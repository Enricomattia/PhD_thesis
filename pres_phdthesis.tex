\documentclass[usenames,dvipsnames,aspectratio=169,11pt]{beamer}
%handout,
%aspectratio=43

\usepackage{prespreamble}
\usepackage{natbib}
\usepackage[normalem]{ulem}
\usepackage{colortbl, xcolor}
\definecolor{RoyalBlue}{rgb}{0.19, 0.55, 0.91}
\linespread{1}
\usepackage{mathrsfs}
\usepackage{transparent}

\let\oldthebibliography=\thebibliography
\renewcommand{\thebibliography}[1]{
    \oldthebibliography{#1}
    \setlength{\itemsep}{2pt}
    \tiny
}

%last slide

\usefonttheme{serif}
\usefonttheme{professionalfonts}
\usepackage{lmodern}
\usepackage{csquotes}

\begin{document}

\begin{frame}[noframenumbering,plain]
	\maketitle

\end{frame}

\begin{frame}\frametitle{Introduction}

	I study individual behaviour, information processing, and resource allocation. \pause

	\vfill

	\begin{wideenumerate}

		\item \textbf{Universalisation}: what would happen were everyone to behave like me?

		\footnotesize{\citep{algerHomoMoralisPreference2013,laffontMacroeconomicConstraintsEconomic1975,kant1785grundlegung,roemer2019cooperate,vanleeuwenEstimatingSocialPreferences2024}}
		\pause
		\vfill

		\normalsize
		\item \textbf{Meritocracy}: an allocation rule that rewards more meritorious individuals.

		\footnotesize{\citep{cappelenFairUnfairIncome2020,fleurbaey2008fairness,kaganGeometryDesert2014,sandelTyrannyMeritWhat2020,senMeritJustice2000}} \pause
		\vfill

		\normalsize
		\item \textbf{Belief-dependent tastes}: individuals who \textquote{like} having specific beliefs.

		\footnotesize{\citep{benabou2016mindful,geanakoplosPsychologicalGamesSequential1989,golmanInformationAvoidance2017,leggPragmatism2024}}
	\end{wideenumerate}


\end{frame}

\begin{frame}{1. A Foundation for Universalisation in Games}
	An individual \( i \) in a two-player game:

	\vfill

	\begin{wideitemize}
		\item chooses a mixed action \( \alpha_i \);
		\item has a belief about opponent's actions \( p_i \);
		\item \textquote{universalises} his action \( \alpha_i \) to an opponent action \( T [\alpha_{i} ] = \alpha_{-i} \). \pause
	\end{wideitemize}

	\vfill

	A \textbf{universalisation preference} is

	\vfill

	\[
		\begin{aligned}
			U_i(\alpha_i) = \invisible{(1-\kappa)} \underbrace{\sum_{a_i, a_{-i}} \alpha_i(a_i) \textcolor{red}{p_{i}(a_{-i})} u_i(a_i, a_{-i})}_{\text{\textcolor{red}{Subjective Expected Utility}}} \invisible{\: + \: \: \kappa \underbrace{\sum_{a_i, a_{-i}} \alpha_i(a_i) \textcolor{blue}{T [ \alpha_i ](a_{-i})} u_i(a_i, a_{-i})}_{\text{\textcolor{blue}{Universalisation}}}}
		\end{aligned}
	\]

\end{frame}

\begin{frame}[noframenumbering]{1. A Foundation for Universalisation in Games}
	An individual \( i \) in a two-player game:

	\vfill

	\begin{wideitemize}
		\item chooses a mixed action \( \alpha_i \);
		\item has a belief about opponent's actions \( p_i \);
		\item \textquote{universalises} his action \( \alpha_i \) to an opponent action \( T [\alpha_{i} ] = \alpha_{-i} \).
	\end{wideitemize}

	\vfill

	A \textbf{universalisation preference} is

	\vfill

	\[
		\begin{aligned}
			U_i(\alpha_i) = \invisible{(1-\kappa)} \underbrace{\sum_{a_i, a_{-i}} \alpha_i(a_i) \textcolor{red}{p_{i}(a_{-i})} u_i(a_i, a_{-i})}_{\text{\textcolor{red}{Subjective Expected Utility}}} \: + \: \: \invisible{\kappa} \underbrace{\sum_{a_i, a_{-i}} \alpha_i(a_i) \textcolor{blue}{T [ \alpha_i ](a_{-i})} u_i(a_i, a_{-i})}_{\text{\textcolor{blue}{Universalisation}}} .
		\end{aligned}
	\]

\end{frame}

\begin{frame}[noframenumbering]{1. A Foundation for Universalisation in Games}
	An individual \( i \) in a two-player game:

	\vfill

	\begin{wideitemize}
		\item chooses a mixed action \( \alpha_i \);
		\item has a belief about opponent's actions \( p_i \);
		\item \textquote{universalises} his action \( \alpha_i \) to an opponent action \( T [\alpha_{i} ] = \alpha_{-i} \).
	\end{wideitemize}

	\vfill

	A \textbf{universalisation preference} is

	\vfill

	\[
		\begin{aligned}
			U_i(\alpha_i) = (1-\kappa) \underbrace{\sum_{a_i, a_{-i}} \alpha_i(a_i) \textcolor{red}{p_{i}(a_{-i})} u_i(a_i, a_{-i})}_{\text{\textcolor{red}{Subjective Expected Utility}}} \: + \: \: \kappa \underbrace{\sum_{a_i, a_{-i}} \alpha_i(a_i) \textcolor{blue}{T [ \alpha_i ](a_{-i})} u_i(a_i, a_{-i})}_{\text{\textcolor{blue}{Universalisation}}} .
		\end{aligned}
	\]

\end{frame}

\begin{frame}{1. Universalisation - Discussion and Results}
	\underline{\textit{Main Result:}}

	I axiomatise universalisation studying preferences over mixed actions. \pause

	\vfill

	Independence is only satisfied between actions universalised \( T \)-\textquote{equivalently}. \pause

	\vfill

	Specifying the function \( T \) allows to study different types of universalisation. \pause

	\vfill

	I introduce \textbf{Equal sacrifice universalisation}.

\end{frame}

\begin{frame}{2. Meritocracy as an End and as a Means}
	Individuals in an economy have preferences over outcomes \( \succsim \). \pause

	\vfill

	A preference \( \succsim \) is more \textquote{meritorious} than \( \succsim^{\prime} \) if

	\[ \succsim M \succsim^{\prime} . \] \pause

	An outcome \( x \) is more \textquote{rewarding} than \( x^{\prime} \) to individual \( i \) if

	\[ x R_i x^{\prime} .\] \pause

	A social choice function \( f \) maps preference profiles to outcomes. \pause

	\vfill

	The function \( f \) is \textbf{meritocratic} if more meritorious preferences are rewarded more.

\end{frame}

\begin{frame}{2. Meritocracy - Discussion and Results}
	A meritocratic social choice function represents \textbf{meritocracy as an end}. \pause

	\vfill

	When individuals are in a game their preferences dictate their actions.

	\vfill

	I then define meritocracy based on behaviour in a game: \textbf{meritocracy as a means}. \pause

	\vfill

	I show that meritocracy as an end and as a means are equivalent. \pause
	\vfill

	\underline{\textit{Main Result:}}

	I introduce and discuss \textbf{Pareto Meritocracy} and \textbf{Proportional Meritocracy}.

\end{frame}

\begin{frame}{3. Identifying Belief-dependent Preferences}

	An individual in a decision problem:

	\vfill

	\begin{itemize}
		\item has a belief \( p \) over uncertain states in \( \mathcal{S} \); \pause
		\item observes a likelihood function \( \ell_S \) putting weight only on states in \( S \subseteq \mathcal{S} \); \pause
		\item the Bayesian update of \( p \) given \( \ell_S \) is \( p_{\ell_S} \); \pause
		\item chooses an act \( f \) mapping states to outcomes. \pause
	\end{itemize}

	\vfill

	A preference with \textbf{belief-dependent tastes} is
	\vfill

	\[
		U (f ; \ell_S)= \underbrace{\sum_s p_{\textcolor{red}{\ell_S}} (s) u\left(f_s ; \textcolor{red}{\ell_S} \right)}_{\text{\textcolor{red}{Belief-dependent utility}}} \invisible{+ \: \: \: \alpha_{\ell_S} \underbrace{\sum_s p_{\textcolor{blue}{\ell^{*}_{S}}} (s) u\left(f_s ; \textcolor{blue}{\ell^{*}_{S}} \right)}_{\text{\textcolor{blue}{Utility with distorted likelihood}}}}
	\]


\end{frame}

\begin{frame}[noframenumbering]{3. Identifying Belief-dependent Preferences}

	An individual in a decision problem:

	\vfill

	\begin{itemize}
		\item has a belief \( p \) over uncertain states in \( \mathcal{S} \);
		\item observes a likelihood function \( \ell_S \) putting weight only on states in \( S \subseteq \mathcal{S} \);
		\item the Bayesian update of \( p \) given \( \ell_S \) is \( p_{\ell_S} \);
		\item chooses an act \( f \) mapping states to outcomes.
	\end{itemize}

	\vfill

	A preference with \textbf{belief-dependent tastes} is
	\vfill

	\[
		U (f ; \ell_S)= \underbrace{\sum_s p_{\textcolor{red}{\ell_S}} (s) u\left(f_s ; \textcolor{red}{\ell_S} \right)}_{\text{\textcolor{red}{Belief-dependent utility}}} \: \: + \: \: \invisible{\alpha_{\ell_S}} \underbrace{\sum_s p_{\textcolor{blue}{\ell^{*}_{S}}} (s) u\left(f_s ; \textcolor{blue}{\ell^{*}_{S}} \right)}_{\text{\textcolor{blue}{Utility with distorted likelihood}}}
	\]


\end{frame}

\begin{frame}[noframenumbering]{3. Identifying Belief-dependent Preferences}

	An individual in a decision problem:

	\vfill

	\begin{itemize}
		\item has a belief \( p \) over uncertain states in \( \mathcal{S} \);
		\item observes a likelihood function \( \ell_S \) putting weight only on states in \( S \subseteq \mathcal{S} \);
		\item the Bayesian update of \( p \) given \( \ell_S \) is \( p_{\ell_S} \);
		\item chooses an act \( f \) mapping states to outcomes.
	\end{itemize}

	\vfill

	A preference with \textbf{belief-dependent tastes} is
	\vfill

	\[
		U (f ; \ell_S)= \underbrace{\sum_s p_{\textcolor{red}{\ell_S}} (s) u\left(f_s ; \textcolor{red}{\ell_S} \right)}_{\text{\textcolor{red}{Belief-dependent utility}}} \: \: + \: \: \alpha_{\ell_S} \underbrace{\sum_s p_{\textcolor{blue}{\ell^{*}_{S}}} (s) u\left(f_s ; \textcolor{blue}{\ell^{*}_{S}} \right)}_{\text{\textcolor{blue}{Utility with distorted likelihood}}}
	\]

	\vfill

	where \( \textcolor{blue}{\ell^{*}_{S}} \) maximises \( u \) under the best possible outcome.

\end{frame}

\begin{frame}{3. Belief-dependent tastes - Discussion and Results}
	Belief-dependent tastes constitute a significant departure from \cite{savageFoundationsStatistics1972}. \pause

	\vfill

	I introduce a novel choice-theoretic primitive: \textbf{contingent menus}. \pause

	\vfill

	\underline{\textit{Main Result:}}

	I axiomatise belief-dependent tastes studying preferences over contingent menus. \pause

	\vfill

	Belief-dependent tastes \textbf{imply} non-Bayesian updating.

	\vfill

	Particular commitment devices enhance welfare.
\end{frame}

\begin{frame}{A methodological takeaway}

	The concepts under study explain behaviour in economically relevant settings, but:

	\vfill

	\begin{wideitemize}
		\item constitute significant departures from theoretical benchmarks;
		\item lack of focus on the logical relationships between these and benchmarks.
	\end{wideitemize} \pause

	\vfill

	I attempt to integrate novel concepts in a logically consistent corpus of knowledge.

	\vfill

	I argue that generalisation to accommodate novel ideas opens unforeseen routes.

\end{frame}

\begin{frame}[noframenumbering,plain]

	\frametitle{References}

	%\nocite{*}
	\bibliography{chapters/references}
	\bibliographystyle{apacite}


\end{frame}

\end{document}