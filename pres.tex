\documentclass[usenames,dvipsnames,aspectratio=169,11pt,handout]{beamer}
%handout,
%aspectratio=43

\usepackage{prespreamble}
\usepackage{natbib}
\usepackage[normalem]{ulem}
\usepackage{colortbl, xcolor}
\definecolor{RoyalBlue}{rgb}{0.19, 0.55, 0.91}
\linespread{1}
\usepackage{mathrsfs}

\let\oldthebibliography=\thebibliography
\renewcommand{\thebibliography}[1]{
    \oldthebibliography{#1}
    \setlength{\itemsep}{2pt}
    \tiny
}


\usefonttheme{serif}
\usefonttheme{professionalfonts}
\usepackage{lmodern}
\usepackage{csquotes}

\begin{document}

\begin{frame}[noframenumbering,plain]
	\maketitle

\end{frame}

\begin{frame}\frametitle{Introduction}

	I study individual behaviour, information processing, and resource allocation. \pause

	\vfill

	\begin{wideenumerate}

		\item \textbf{Universalisation}: what would happen were everyone to behave like me?

		\citep{algerHomoMoralisPreference2013,laffontMacroeconomicConstraintsEconomic1975,kant1785grundlegung,roemer2019cooperate}
		\vfill \pause

		\item \textbf{Meritocracy}: allocation rule rewarding more meritorious individuals.

		\citep{cappelenFairUnfairIncome2020,fleurbaey2008fairness,kaganGeometryDesert2014,senMeritJustice2000} \pause
		\vfill

		\item \textbf{Belief-dependent tastes}: tastes over beliefs.

		\citep{benabou2016mindful,brunnermeierOptimalExpectations2005,golmanInformationAvoidance2017}
	\end{wideenumerate}


\end{frame}

\begin{frame}{1. A Foundation for Universalisation in Games}
	An individual \( i \) in a game:

	\vfill

	\begin{wideitemize}
		\item chooses a mixed action \( \alpha_i \);
		\item has a belief about opponent's actions \( p_i \);
		\item \textquote{universalises} his action to an opponent action with the function \( T \).
	\end{wideitemize}

	\vfill

	A \textbf{universalisation preference} is

	\vfill

	\[
		\begin{aligned}
			U_i(\alpha_i) = (1-\kappa) \underbrace{\sum_{a_i, a_{-i}} \alpha_i(a_i) p_{i}(a_{-i}) u_i(a_i, a_{-i})}_{\text{Subjective Expected Utility}} + \kappa \underbrace{\sum_{a_i, a_{-i}} \alpha_i(a_i) T [ \alpha_i ](a_{-i}) u_i(a_i, a_{-i})}_{\text{Universalisation}} .
		\end{aligned}
	\]

\end{frame}

\begin{frame}{1. Universalisation - Discussion}
	I axiomatise universalisation preferences studying preferences over mixed actions.

	\vfill

	Independence is only satisfied only between actions universalised \textquote{equivalently}.

	\vfill

	Specifying the function \( T \) allows to study different types of universalisation.

	\vfill

	I introduce \textbf{Equal sacrifice universalisation}.

\end{frame}

\begin{frame}{2. Meritocracy as an End and as a Means}
	Individual \( i \) has preference over outcomes \( \succsim_i \).

	\vfill

	A preference \( \succsim_i \) is more \textquote{meritorious} than \( \succsim_i^{\prime} \) if \( \succsim_i M \succsim_i^{\prime} \).

	\vfill

	An outcome \( x \) is more \textquote{rewarding} than \( x^{\prime} \) to individual \( i \) if \( x R_i x^{\prime} \).

	\vfill

	A social choice function \( f \) maps preference profiles to outcomes.

	\vfill

	The function \( f \) is \textbf{meritocratic} if more meritorious preferences are rewarded more.

\end{frame}

\begin{frame}{Chapter 2. Meritocracy - Discussion}
	A meritocratic social choice function represents \textit{meritocracy as an end}.

	\vfill

	When individuals are in a game their preferences dictate their actions.

	\vfill

	One could then define meritocracy based on behaviour in a game.

	\vfill

	The latter is \textit{meritocracy as a means}.

	\vfill

	I show that meritocracy as an end and as a means are equivalent.
	\vfill

	I discuss \textbf{Pareto Meritocracy} and \textbf{Proportional Meritocracy}.

\end{frame}

\begin{frame}{3. Identifying Belief-dependent Preferences}

	An individual in a decision problem:

	\vfill

	\begin{wideitemize}
		\item has a prior belief \( p \) over uncertain states in \( \mathcal{S} \);
		\item observes a likelihood function \( \ell_S \) putting weight only on states in \( S \subseteq \mathcal{S} \);
		\item the bayesian update of \( p \) given \( \ell_S \) is \( p_{\ell_S} \);
		\item chooses an act \( f \) mapping states to outcomes.
	\end{wideitemize}

	\vfill

	A preference with \textbf{belief-dependent tastes} is
	\vfill

	\[
		\mathcal{U}(f ; \ell_S)= \underbrace{\sum_s p_{\ell_S}(s) u\left(f_s ; \ell_S \right)}_{\text{Belief-dependent utility}} + \: \alpha_{\ell_S} \underbrace{\sum_s p_{\ell^{*}_{S}} (s) u\left(f_s ; \ell^{*}_{S} \right)}_{\text{Utility with distorted likelihood}}.
	\]


\end{frame}

\begin{frame}{Chapter 3. Belief-dependent tastes - Discussion}
	Belief-dependent tastes constitute a significant departure from \cite{savageFoundationsStatistics1972}.

	\vfill

	I show how to identify belief-dependent tastes from choices of \textbf{contingent menus}.
\end{frame}

\begin{frame}{A methodological takeaway}
	Novel concepts require \textbf{conceptual} understanding.
\end{frame}

\begin{frame}[noframenumbering,plain]

	\frametitle{References}

	%\nocite{*}
	\bibliography{chapters/references}
	\bibliographystyle{apacite}


\end{frame}

\end{document}