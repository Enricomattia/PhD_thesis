\chapter{Appendix for Chapter \ref{ch:meritocracy}}
\label{app:meritocracy}

\section{Proofs}\label{sec:proofsmeritocracy}

\begin{implproof}
	First, I show that if a mechanism implements a meritocratic social choice function, it is meritocratic with respect to merit criteria \( A_i \) which agree with \( M \) for each \( i \). Assume that \( \mu \) implements a meritocratic \( f \), that is, for each preference profile \( \succsim \) and for every Nash equilibrium strategy profile \( a^{*} ( \succsim ) \) it holds that

	\[
		g ( a^{*} ( \succsim ) ) = f ( \succsim ).
	\]

	Since \( f \) is meritocratic, for each individual \( i \) and profile \( \succsim_{-i} \), for all preferences \( \succsim_i, \succsim'_i \), it holds that

	\[
		\succsim_i M \succsim'_i \implies f (\succsim_i, \succsim_{-i}) R_i f (\succsim'_i, \succsim_{-i} ).
	\]

	Equivalently,

	\[
		\succsim_i M \succsim'_i \implies g ( a_i^*(\succsim_i, \succsim_{-i}), a_{-i}^*(\succsim) ) R_i g ( a_i^*(\succsim_i^{\prime}, \succsim_{-i}), a_{-i}^*(\succsim) ),
	\]

	and therefore the mechanism is meritocratic with respect to some merit criteria such that

	\[ ( a_i^*(\succsim_i, \succsim_{-i}), a_{-i}^*(\succsim_i, \succsim_{-i}) ) A_i ( a_i^*(\succsim_i^{\prime}, \succsim_{-i}), a_{-i}^*(\succsim_i^{\prime}, \succsim_{-i}) ) .\]

	I show that each merit criterion \( A_i \) must agree with \( M \). Suppose, for the sake of contradiction, that for some individual \( i \) and some fixed \( a_{-i} \) the criteria \( A_i \) and \( M \) do not agree. Then there exist two actions \( a_i, a'_i \) and corresponding preferences \( \succsim_i, \succsim'_i \) such that

	\[
		\succsim_i M \succsim'_i
		\quad \text{but} \quad
		( a_i^*(\succsim_i^{\prime}, \succsim_{-i}), a_{-i}^*(\succsim_i^{\prime}, \succsim_{-i}) ) A_i ( a_i^*(\succsim_i, \succsim_{-i}), a_{-i}^*(\succsim_i, \succsim_{-i}) ) .
	\]

	Then, the mechanism would not be implementing the meritocratic \( f \), as for the mechanism to be meritocratic it must be the case that

	\[
		g (a_i^*(\succsim_i^{\prime}, \succsim_{-i}), a_{-i}^*(\succsim_i^{\prime}, \succsim_{-i})) R_i g (a_i^*(\succsim_i, \succsim_{-i}), a_{-i}^*(\succsim_i, \succsim_{-i}) ) ,
	\]

	but

	\[ f (\succsim_i, \succsim_{-i}) R_i f (\succsim'_i, \succsim_{-i} ) . \]

	Conversely, assume that \( \mu \) is meritocratic with respect to some criteria \( A_i \). Then, for each individual \( i \) and profile \( \succsim_{-i} \), for all preferences \( \succsim_i, \succsim'_i \), for each Nash equilibrium profile \( a^{*} ( \succsim ) \), it holds that

	\[
		( a_i^*(\succsim_i, \succsim_{-i}), a_{-i}^*(\succsim_i, \succsim_{-i}) ) A_i ( a_i^*(\succsim_i^{\prime}, \succsim_{-i}), a_{-i}^*(\succsim_i^{\prime}, \succsim_{-i}) )
	\]

	\[ \implies \]

	\[
		g ( a_i^*(\succsim_i, \succsim_{-i}), a_{-i}^*(\succsim_i, \succsim_{-i}) ) R_i g ( a_i^*(\succsim_i^{\prime}, \succsim_{-i}), a_{-i}^*(\succsim_i^{\prime}, \succsim_{-i}) ) .
	\]

	Then, such mechanism implements a meritocratic social choice function \( f \) with respect to some criterion \( M \). By an argument similar to the one of the previous paragraph, it follows that each \( A_i \) and \( M \) agree.
\end{implproof}

\begin{paretoproof}
	Fix an individual \( i \) and consider two preference relations \( \succsim_i \) and \( \succsim'_i \), each having a unique maximal element \( x^{*}(\succsim_i) \) and \( x^{*}(\succsim'_i) \) in \( X \), respectively. Fix any profile \( a_{-i} \) and consider actions \( a_i, a'_i \) such that

	\[
		g(a_i,a_{-i}) = x^{*}(\succsim_i) \quad \text{and} \quad g(a'_i,a_{-i}) = x^{*}(\succsim'_i).
	\]

	By the agreement condition between the merit criteria \( A_i \) and \( M \), it must hold that

	\[
		(a_i,a_{-i}) A_i (a'_i,a_{-i}) \iff \succsim_i M \succsim'_i.
	\]

	Since \( A_i \) satisfies \usename{axn:pareto}, substituting \( x^{*}(\succsim_i) \) and \( x^{*}(\succsim'_i) \) delivers

	\[
		(a_i,a_{-i}) A_i (a'_i,a_{-i}) \iff \Bigl[ x^{*}(\succsim_i) \, R_j \, x^{*}(\succsim'_i) \text{ for all } j \text{ and } x^{*}(\succsim_i) \, P_j \, x^{*}(\succsim'_i) \text{ for some } j \Bigr]\,.
	\]

	By the agreement condition, it follows that

	\[
		\succsim_i M \succsim'_i \iff \Bigl[ x^{*}(\succsim_i) \, R_j \, x^{*}(\succsim'_i) \text{ for all } j \text{ and } x^{*}(\succsim_i) \, P_j \, x^{*}(\succsim'_i) \text{ for some } j \Bigr]\,.
	\]
\end{paretoproof}

\begin{propproof}
	I start with item 1. I first show the necessity of assumptions. Suppose the mechanism is a \usename{def:prop}.

	\emph{Meritocracy.} Define the merit criterion so that \((\ell_i,\ell_{-i}) A_i (\ell_i',\ell_{-i})\) means \(\ell_i\ge\ell_i'\). Then by the monotonicity of \(\alpha_i\), it holds that \(\alpha_i(\ell_i,\ell_{-i}) \ge \alpha_i(\ell_i',\ell_{-i})\) whenever \(\ell_i\ge\ell_i'\). Hence

	\[
		g_i(\ell_i,\ell_{-i})
		\;=\;\alpha_i(\ell_i,\ell_{-i})\,c^{-1} \left(\sum_j \ell_j\right)
		\;\;\ge\;\;
		\alpha_i(\ell_i',\ell_{-i})\,c^{-1} \left(\sum_j \ell_j\right)
		\;=\;g_i(\ell_i',\ell_{-i}),
	\]

	which means a more meritorious action yields a strictly preferred outcome for \(i\). Thus the mechanism is meritocratic.

	\emph{\usename{axn:welfrew}.}
	In a \usename{def:prop}, each individual \(i\)'s consumption is

	\[ \alpha_i(\ell)\,c^{-1} \left( \sum_j\ell_j \right) . \]

	Since each individual preferences are increasing in consumption, \usename{axn:welfrew} is satisfied.

	\emph{\usename{axn:mono}.}
	Since \(\alpha_i\) is monotonic in \(\ell_i\), if \(\ell_i\ge\ell_i'\), holding \(\ell_{-i}\) fixed,  then \(\alpha_i(\ell_i,\ell_{-i}) \ge \alpha_i(\ell_i',\ell_{-i})\). Thus \((\ell_i,\ell_{-i}) A_i (\ell_i',\ell_{-i})\).

	\emph{\usename{axn:scale}.}
	Since \(\alpha_i\) is homogeneous of degree zero, scaling \(\ell \) to \( \lambda \ell \) does not change the values of \(\alpha_i(\cdot)\). Hence if \((\ell_i,\ell_{-i}) A_i (\ell_i',\ell_{-i})\),	the same ordering persists after multiplying all labour inputs by any \(\lambda>0\).

	I now show that the assumptions imply the mechanism is a \usename{def:prop}.

	\emph{Feasibility.}
	Maximal total consumption is

	\[ c^{-1} \left( \sum_j \ell_j \right) .\]

	Since preferences are increasing in consumption, a no-waste argument implies

	\[ \sum_{i=1}^n g_i(\ell)= c^{-1} \left( \sum_j \ell_j \right) .\]

	Define the shares

	\[
		\alpha_i(\ell)
		\;:=\;
		\frac{\,g_i(\ell)\,}
		{c^{-1} \left( \sum_j\ell_j \right)}.
	\]

	Then 	\( \sum_i \alpha_i(\ell)=1\). I prove that \(\alpha_i(\ell)\) is monotonic in \(\ell_i\), holding \(\ell_{-i}\) fixed, and homogeneous of degree zero.

	\emph{Monotonicity.}
	By \usename{axn:mono'}, \(\ell_i\ge\ell_i'\) implies \((\ell_i,\ell_{-i}) A_i (\ell_i',\ell_{-i})\). Because \(\mu\) is meritocratic and \(R_i=\succsim_i\), by \usename{axn:welfrew}, it follows that \(g_i(\ell_i,\ell_{-i})\) is strictly better, i.e., a larger consumption, than \(g_i(\ell_i',\ell_{-i})\). Hence
	\(\alpha_i(\ell_i,\ell_{-i}) \ge \alpha_i(\ell_i',\ell_{-i})\). So \(\alpha_i\) is non-decreasing in \(\ell_i\).

	\emph{Scale-Invariance.}
	The ordering \((\ell_i,\ell_{-i}) A_i (\ell_i',\ell_{-i})\) implies \((\lambda\ell_i,\lambda\ell_{-i}) A_i (\lambda\ell_i',\lambda\ell_{-i})\) by \usename{axn:scale}. Since \(\ell_i\ge\ell_i'\iff(\ell_i,\ell_{-i})A_i(\ell_i',\ell_{-i})\), this ordering persists under scaling, and hence \(\alpha_i(\lambda\ell)=\alpha_i(\ell)\). Thus \(\alpha_i\) is homogeneous of degree zero.

	Together, these conditions imply the mechanism

	\[
		g_i(\ell)
		\;=\;
		\alpha_i(\ell)\;\;c^{-1} \left(\sum_j \ell_j \right)
	\]

	is a \usename{def:prop}. This completes the proof of item 1.

	I now prove item 2. I first show that if

	\[
		g_i(\ell)
		\;=\;
		\frac{\ell_i}{\sum_j \ell_j}\;c^{-1} \left(\sum_j \ell_j \right),
	\]

	then \usename{axn:mono'} holds. As before, one checks that \(\sum_i \alpha_i(\ell)=1\), \(\alpha_i(\ell)\) is strictly increasing in \(\ell_i\), and \(\alpha_i(\lambda\ell)=\alpha_i(\ell)\). Defining \((\ell_i,\ell_{-i}) A_i (\ell_i',\ell_{-i})\) to mean \(\ell_i\ge \ell_i'\) satisfies \usename{axn:mono'}, because

	\[
		\ell_i\;\ge\;\ell_i'
		\;\;\Longleftrightarrow\;\;
		\frac{\ell_i}{\sum_j\ell_j}
		\;\ge\;\frac{\ell_i'}{\sum_j\ell_j}
		\;\Longleftrightarrow\;\alpha_i(\ell)\;\ge\;\alpha_i(\ell').
	\]

	I now show that the assumptions imply item 2. Assume the mechanism is meritocratic, satisfies \usename{axn:welfrew}, \usename{axn:mono'} and \usename{axn:scale}. By part 1., it is already a \usename{def:prop}. I must show that

	\[ \alpha_i(\ell)=\frac{\ell_i}{\sum_j \ell_j} . \]

	Since \usename{axn:mono'} means

	\[
		\ell_i\;\ge\;\ell_i'
		\quad\Longleftrightarrow\quad
		(\ell_i,\ell_{-i}) \;A_i\; (\ell_i',\ell_{-i}),
	\]

	meritocracy plus \usename{axn:welfrew} implies \(\ell_i\ge\ell_i'\iff\alpha_i(\ell)\ge\alpha_i(\ell')\). Also, \usename{axn:scale} forces \(\alpha_i(\lambda\ell)=\alpha_i(\ell)\). Hence \(\alpha_i(\ell)\) depends only on the ratio \(\ell_i/\sum_j\ell_j\). That is,

	\[
		\alpha_i(\ell)
		\;=\;
		F \left( \frac{\ell_i}{\sum_j \ell_j} \right)
	\]

	for some strictly increasing function \(F\). Since \(\sum_i \alpha_i(\ell)=1\), a standard argument \citep[ch. 2 Th. 2]{aczelLecturesFunctionalEquations1966} yields \(F(x)=x\). Thus

	\[
		\alpha_i(\ell)
		\;=\;
		\frac{\ell_i}{\sum_j\ell_j}.
	\]

	Lastly, I show agreement with the merit criterion \( M \). The argument proceeds by contradiction: if \(\mathrm{MRS}_{\succsim_i}\) is not uniformly smaller than \(\mathrm{MRS}_{\succsim_i^{\prime}}\), the agreement fails. Suppose there exist preferences \(\succsim_i\) and \(\succsim_i^{\prime}\) such that \(\mathrm{MRS}_{\succsim_i}(\ell_i,y_i) < \mathrm{MRS}_{\succsim_i^{\prime}}(\ell_i,y_i)\) in some regions, but \(\mathrm{MRS}_{\succsim_i}(\ell_i,y_i) \ge \mathrm{MRS}_{\succsim_i^{\prime}}(\ell_i,y_i)\) in others. I show this violates agreement.

	In the region where \(\succsim_i\) has \(\mathrm{MRS}_{\succsim_i} < \mathrm{MRS}_{\succsim_i^{\prime}}\), it is easier to make \(\succsim_i\) better off with fewer consumption units for the same labour.  One can select a labour level \(\ell_i\) at which

	\[
		g_i(\ell_i,\ell_{-i})
		\;=\;
		\frac{\ell_i}{\sum_j\ell_j}\,c^{-1} \left( \sum_j\ell_j \right)
	\]

	is strictly preferred by \(\succsim_i\) to the outcome from a smaller labour \(\ell_i'\). In the region where \(\succsim_i\) has \(\mathrm{MRS}_{\succsim_i}\) greater than \(\mathrm{MRS}_{\succsim_i^{\prime}}\), one can choose a smaller labour \(\ell_i'\) that yields an outcome \(g_i(\ell_i',\ell_{-i})\) preferred by \(\succsim_i^{\prime}\) to the outcome of the bigger labour \(\ell_i\). Hence one obtains

	\[
		g(\ell_i,\ell_{-i})
		\; \succsim_i\;
		g(\ell_i',\ell_{-i})
		\quad\text{and}\quad
		g(\ell_i',\ell_{-i})
		\; \succsim_i^{\prime}\;
		g(\ell_i,\ell_{-i}) .
	\]

	However, agreement demands

	\[
		\ell_i \;\ge\;\ell_i'
		\quad\Longleftrightarrow\quad
		\succsim_i \;M\; \succsim_i^{\prime}.
	\]

	But also reversing the roles (\(\ell_i'\ge \ell_i\)) would force
	\(\succsim_i^{\prime}M \succsim_i\), and thus I reached a contradiction.

\end{propproof}

\bibliographystyle{apacite}  % or another  style
\bibliography{references} % .bib file goes in ./bib/
