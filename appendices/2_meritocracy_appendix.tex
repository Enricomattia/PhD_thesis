\chapter{Appendix for Chapter \ref{ch:meritocracy}}
\label{app:meritocracy}

\section{Proofs}\label{sec:proofsmeritocracy}

\begin{implproof}
	First, I show that if a mechanism implements a meritocratic social choice function, it is meritocratic with respect to merit criteria \( \geq_i \) which agree with \( \geq_{\mathcal{R}} \) for each \( i \). Assume that \( M \) implements a meritocratic \( f \), that is, for each preference profile \( R \) and for every Nash equilibrium strategy profile \( a^{*} ( R ) \) it holds that

	\[
		g ( a^{*} ( R ) ) = f ( R ).
	\]

	Since \( f \) is meritocratic, for each individual \( i \) and profile \( R_{-i} \), for all preferences \( R_i, R'_i \), it holds that

	\[
		R_i \geq_{\mathcal{R}} R'_i \implies f (R_i, R_{-i}) \geq_i f (R'_i, R_{-i} ).
	\]

	Equivalently,

	\[
		R_i \geq_{\mathcal{R}} R'_i \implies g ( a_i^*(R_i, R_{-i}), a_{-i}^*(R) ) \geq_i g ( a_i^*(R_i', R_{-i}), a_{-i}^*(R) ),
	\]

	and therefore the mechanism is meritocratic with respect to some merit criteria such that

	\[ ( a_i^*(R_i, R_{-i}), a_{-i}^*(R_i, R_{-i}) ) \geq_{A_i} ( a_i^*(R_i', R_{-i}), a_{-i}^*(R_i', R_{-i}) ) .\]

	I show that each merit criterion \( \geq_{A_i} \) must agree with \( \geq_{\mathcal{R}} \). Suppose, for the sake of contradiction, that for some individual \( i \) and some fixed \( a_{-i} \) the criteria \( \geq_{A_i} \) and \( \geq_{\mathcal{R}} \) do not agree. Then there exist two actions \( a_i, a'_i \) and corresponding preferences \( R_i, R'_i \) such that

	\[
		R_i \geq_{\mathcal{R}} R'_i
		\quad \text{but} \quad
		( a_i^*(R_i', R_{-i}), a_{-i}^*(R_i', R_{-i}) ) \geq_{A_i} ( a_i^*(R_i, R_{-i}), a_{-i}^*(R_i, R_{-i}) ) .
	\]

	Then, the mechanism would not be implementing the meritocratic \( f \), as for the mechanism to be meritocratic it must be the case that

	\[
		g (a_i^*(R_i', R_{-i}), a_{-i}^*(R_i', R_{-i})) \geq_i g (a_i^*(R_i, R_{-i}), a_{-i}^*(R_i, R_{-i}) ) ,
	\]

	but

	\[ f (R_i, R_{-i}) \geq_i f (R'_i, R_{-i} ) . \]

	Conversely, assume that \( M \) is meritocratic with respect to some criteria \( \geq_{A_i} \). Then, for each individual \( i \) and profile \( R_{-i} \), for all preferences \( R_i, R'_i \), for each Nash equilibrium profile \( a^{*} ( R ) \), it holds that

	\[
		( a_i^*(R_i, R_{-i}), a_{-i}^*(R_i, R_{-i}) ) \geq_{A_i} ( a_i^*(R_i', R_{-i}), a_{-i}^*(R_i', R_{-i}) )
	\]

	\[ \implies \]

	\[
		g ( a_i^*(R_i, R_{-i}), a_{-i}^*(R_i, R_{-i}) ) \geq_i g ( a_i^*(R_i', R_{-i}), a_{-i}^*(R_i', R_{-i}) ) .
	\]

	Then, such mechanism implements a meritocratic social choice function \( f \) with respect to some criterion \( \geq_{\mathcal{R}} \). By an argument similar to the one of the previous paragraph, it follows that each \( \geq_{A_i} \) and \( \geq_{\mathcal{R}} \) agree.
\end{implproof}

\begin{paretoproof}
	Fix an individual \( i \) and consider two preference relations \( R_i \) and \( R'_i \), each having a unique maximal element \( x^{*}(R_i) \) and \( x^{*}(R'_i) \), respectively. Fix any profile \( a_{-i} \) and consider actions \( a_i, a'_i \) such that

	\[
		g(a_i,a_{-i}) = x^{*}(R_i) \quad \text{and} \quad g(a'_i,a_{-i}) = x^{*}(R'_i).
	\]

	By the agreement condition between the merit criteria \( \geq_{A_i} \) and \( \geq_{\mathcal{R}} \), it must hold that

	\[
		(a_i,a_{-i}) \geq_{A_i} (a'_i,a_{-i}) \iff R_i \geq_{\mathcal{R}} R'_i.
	\]

	Since \( \geq_{A_i} \) satisfies \usename{axn:pareto}, substituting \( x^{*}(R_i) \) and \( x^{*}(R'_i) \) delivers

	\[
		(a_i,a_{-i}) \geq_{A_i} (a'_i,a_{-i}) \iff \Bigl[ x^{*}(R_i) \, R_j \, x^{*}(R'_i) \text{ for all } j \text{ and } x^{*}(R_i) \, P_j \, x^{*}(R'_i) \text{ for some } j \Bigr]\,.
	\]

	By the agreement condition, it follows that

	\[
		R_i \geq_{\mathcal{R}} R'_i \iff \Bigl[ x^{*}(R_i) \, R_j \, x^{*}(R'_i) \text{ for all } j \text{ and } x^{*}(R_i) \, P_j \, x^{*}(R'_i) \text{ for some } j \Bigr]\,.
	\]
\end{paretoproof}

\begin{propproof}
	I start with item 1. I first show the necessity of assumptions. Suppose the mechanism is a \usename{def:prop}.

	\emph{Meritocracy.} Define the merit criterion so that \((\ell_i,\ell_{-i}) \ge_{A_i} (\ell_i',\ell_{-i})\) means \(\ell_i\ge\ell_i'\). Then by the monotonicity of \(\alpha_i\), it holds that \(\alpha_i(\ell_i,\ell_{-i}) \ge \alpha_i(\ell_i',\ell_{-i})\) whenever \(\ell_i\ge\ell_i'\). Hence

	\[
		g_i(\ell_i,\ell_{-i})
		\;=\;\alpha_i(\ell_i,\ell_{-i})\,c^{-1} \left(\sum_j \ell_j\right)
		\;\;\ge\;\;
		\alpha_i(\ell_i',\ell_{-i})\,c^{-1} \left(\sum_j \ell_j\right)
		\;=\;g_i(\ell_i',\ell_{-i}),
	\]

	which means a more meritorious action yields a strictly preferred outcome for \(i\). Thus the mechanism is meritocratic.

	\emph{\usename{axn:welfrew}.}
	In a \usename{def:prop}, each individual \(i\)'s consumption is

	\[ \alpha_i(\ell)\,c^{-1} \left( \sum_j\ell_j \right) . \]

	Since each individual preferences are increasing in consumption, \usename{axn:welfrew} is satisfied.

	\emph{\usename{axn:mono}.}
	Since \(\alpha_i\) is monotonic in \(\ell_i\), if \(\ell_i\ge\ell_i'\), holding \(\ell_{-i}\) fixed,  then \(\alpha_i(\ell_i,\ell_{-i}) \ge \alpha_i(\ell_i',\ell_{-i})\). Thus \((\ell_i,\ell_{-i}) \ge_{A_i} (\ell_i',\ell_{-i})\).

	\emph{\usename{axn:scale}.}
	Since \(\alpha_i\) is homogeneous of degree zero, scaling \(\ell \) to \( \lambda \ell \) does not change the values of \(\alpha_i(\cdot)\). Hence if \((\ell_i,\ell_{-i}) \ge_{A_i} (\ell_i',\ell_{-i})\),	the same ordering persists after multiplying all labour inputs by any \(\lambda>0\).

	I now show that the assumptions imply the mechanism is a \usename{def:prop}.

	\emph{Total Feasibility.}
	Maximal total consumption is

	\[ c^{-1} \left( \sum_j \ell_j \right) .\]

	Since preferences are increasing in consumption, a no-waste argument implies

	\[ \sum_{i=1}^n g_i(\ell)= c^{-1} \left( \sum_j \ell_j \right) .\]

	Define the shares

	\[
		\alpha_i(\ell)
		\;:=\;
		\frac{\,g_i(\ell)\,}
		{c^{-1} \left( \sum_j\ell_j \right)}.
	\]

	Then 	\( \sum_i \alpha_i(\ell)=1\). I prove that \(\alpha_i(\ell)\) is monotonic in \(\ell_i\), holding \(\ell_{-i}\) fixed, and homogeneous of degree zero.

	\emph{Monotonicity.}
	By \usename{axn:mono'}, \(\ell_i\ge\ell_i'\) implies \((\ell_i,\ell_{-i}) \ge_{A_i} (\ell_i',\ell_{-i})\). Because \(M\) is meritocratic and \(\ge_i=R_i\), by \usename{axn:welfrew}, it follows that \(g_i(\ell_i,\ell_{-i})\) is strictly better, i.e., a larger consumption, than \(g_i(\ell_i',\ell_{-i})\). Hence
	\(\alpha_i(\ell_i,\ell_{-i}) \ge \alpha_i(\ell_i',\ell_{-i})\). So \(\alpha_i\) is non-decreasing in \(\ell_i\).

	\emph{Scale-Invariance.}
	By \usename{axn:scale}, \((\ell_i,\ell_{-i}) \ge_{A_i} (\ell_i',\ell_{-i})\) implies \((\lambda\ell_i,\lambda\ell_{-i}) \ge_{A_i} (\lambda\ell_i',\lambda\ell_{-i})\). Since \(\ell_i\ge\ell_i'\iff(\ell_i,\ell_{-i})\ge_{A_i}(\ell_i',\ell_{-i})\), this ordering persists under scaling, and hence \(\alpha_i(\lambda\ell)=\alpha_i(\ell)\). Thus \(\alpha_i\) is homogeneous of degree zero.

	Together, these conditions imply the mechanism

	\[
		g_i(\ell)
		\;=\;
		\alpha_i(\ell)\;\;c^{-1} \left(\sum_j \ell_j \right)
	\]

	is a \usename{def:prop}. This completes the proof of item 1.

	I now prove item 2. I first show that if

	\[
		g_i(\ell)
		\;=\;
		\frac{\ell_i}{\sum_j \ell_j}\;c^{-1} \left(\sum_j \ell_j \right),
	\]

	then \usename{axn:mono'} holds. As before, one checks that \(\sum_i \alpha_i(\ell)=1\), \(\alpha_i(\ell)\) is strictly increasing in \(\ell_i\), and \(\alpha_i(\lambda\ell)=\alpha_i(\ell)\). Defining \((\ell_i,\ell_{-i}) \ge_{A_i} (\ell_i',\ell_{-i})\) to mean \(\ell_i\ge \ell_i'\) satisfies \usename{axn:mono'}, because

	\[
		\ell_i\;\ge\;\ell_i'
		\;\;\Longleftrightarrow\;\;
		\frac{\ell_i}{\sum_j\ell_j}
		\;\ge\;\frac{\ell_i'}{\sum_j\ell_j}
		\;\Longleftrightarrow\;\alpha_i(\ell)\;\ge\;\alpha_i(\ell').
	\]

	I now show that the assumptions imply item 2. Assume the mechanism is meritocratic, satisfies \usename{axn:welfrew}, \usename{axn:mono'} and \usename{axn:scale}. By part 1., it is already a \usename{def:prop}. I must show that

	\[ \alpha_i(\ell)=\frac{\ell_i}{\sum_j \ell_j} . \]

	Since \usename{axn:mono'} means

	\[
		\ell_i\;\ge\;\ell_i'
		\quad\Longleftrightarrow\quad
		(\ell_i,\ell_{-i}) \;\ge_{A_i}\; (\ell_i',\ell_{-i}),
	\]

	meritocracy plus \usename{axn:welfrew} implies \(\ell_i\ge\ell_i'\iff\alpha_i(\ell)\ge\alpha_i(\ell')\). Also, \usename{axn:scale} forces \(\alpha_i(\lambda\ell)=\alpha_i(\ell)\). Hence \(\alpha_i(\ell)\) depends only on the ratio \(\ell_i/\sum_j\ell_j\). That is,

	\[
		\alpha_i(\ell)
		\;=\;
		F \left( \frac{\ell_i}{\sum_j \ell_j} \right)
	\]

	for some strictly increasing function \(F\). Since \(\sum_i \alpha_i(\ell)=1\), a standard argument \citep[ch. 2 Th. 2]{aczelLecturesFunctionalEquations1966} yields \(F(x)=x\). Thus

	\[
		\alpha_i(\ell)
		\;=\;
		\frac{\ell_i}{\sum_j\ell_j}.
	\]

	Lastly, I show agreement with the merit criterion \( \geq_{\mathcal{R}} \). The argument proceeds by contradiction: if \(\mathrm{MRS}_{R_i}\) is not uniformly smaller than \(\mathrm{MRS}_{R_i'}\), the agreement fails. Suppose there exist preferences \(R_i\) and \(R_i'\) such that \(\mathrm{MRS}_{R_i}(\ell_i,y_i)\le \mathrm{MRS}_{R_i'}(\ell_i,y_i)\) in some regions, but strictly greater in some other region of \((\ell_i,y_i)\). I show this violates agreement.

	In the region where \(R_i\) has \(\mathrm{MRS}_{R_i} < \mathrm{MRS}_{R_i'}\), it is easier to make \(R_i\) better off with fewer consumption units for the same labour.  One can select a labour level \(\ell_i\) at which

	\[
		g_i(\ell_i,\ell_{-i})
		\;=\;
		\frac{\ell_i}{\sum_j\ell_j}\,c^{-1} \left( \sum_j\ell_j \right)
	\]

	is strictly preferred by \(R_i\) to the outcome from a smaller labour \(\ell_i'\). In the region where \(R_i\) has \(\mathrm{MRS}_{R_i}\) greater than \(\mathrm{MRS}_{R_i'}\), one can choose a smaller labour \(\ell_i'\) that yields an outcome \(g_i(\ell_i',\ell_{-i})\) preferred by \(R_i'\) to the outcome of the bigger labour \(\ell_i\). Hence one obtains

	\[
		g(\ell_i,\ell_{-i})
		\; R_i\;
		g(\ell_i',\ell_{-i})
		\quad\text{and}\quad
		g(\ell_i',\ell_{-i})
		\; R_i'\;
		g(\ell_i,\ell_{-i}) .
	\]

	However, agreement demands

	\[
		\ell_i \;\ge\;\ell_i'
		\quad\Longleftrightarrow\quad
		R_i \;\ge_\mathcal{R}\; R_i'.
	\]

	But also reversing the roles (\(\ell_i'\ge \ell_i\)) would force
	\(R_i'\ge_\mathcal{R} R_i\), and thus I reached a contradiction.

\end{propproof}

\bibliographystyle{apacite}  % or another  style
\bibliography{references} % .bib file goes in ./bib/
