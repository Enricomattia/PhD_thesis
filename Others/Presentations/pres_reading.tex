
\documentclass[usenames,dvipsnames,aspectratio=169,11pt, envcountsect]{beamer}
%handout,
%aspectratio=43

\usepackage{Others/prespreamble}

\title {\textsc{Fairness and limited information: Are people Bayesian meritocrats?}}
\author{\textsc{Cappellen, de Haan, Tungodden}}
%\institute{\textsc{Toulouse School of Economics}} 
\date{October 8, 2024}
%\subject{\textsc{Toulouse School of Economics}}


\usepackage{natbib}
\usepackage{colortbl, xcolor}
\definecolor{blue}{rgb}{0.19, 0.55, 0.91}
\linespread{1}
\usepackage{mathrsfs}

% small bibliography
\let\oldthebibliography=\thebibliography
\renewcommand{\thebibliography}[1]{
    \oldthebibliography{#1}
    \setlength{\itemsep}{2pt}
    \tiny
}

\begin{document}

\section{Introduction}

\begin{frame}[noframenumbering,plain]
	\maketitle
	%\begin{figure}[H]
	%\centering
	%\includegraphics[scale=0.4]{Other/Figures/erc.png}
	%\end{figure}

	%\footnotesize{Funding from the European Research Council (ERC) under the European Unions Horizon 2020 research and innovation programme (grant agreement No. 789111 - ERC EvolvingEconomics) is gratefully acknowledged}
\end{frame}
%\frame{\titlepage}

\begin{frame}\frametitle{Introduction}

	\textbf{Q:} Do individuals distort relevant information when allocating resources?

	\vfill

	Two individuals have a performance \( p_i \) and a random factor \( \varepsilon_i \).

	\vfill

	Their earnings are \( x_i = p_i + \varepsilon_i \).

	\vfill

	An impartial spectator has to allocate \( X = x_i + x_j \) between the two individuals:

	\vfill

	\begin{wideitemize}
		\item he might have full information \( p_i, p_j, \varepsilon_i, \varepsilon_j \);
		\item or limited information \( x_i, x_j \).
	\end{wideitemize}

\end{frame}

\begin{frame}\frametitle{Spectator preferences}
	Denote with \( m_i \) what the spectator deems the moral allocation of \( i \).

	\vfill

	He chooses transfer \( y_i \) to maximise

	\[
		U_{\text {spectator }}=- \left(y_i-m_i \right)^2
	\]

	\vfill

	Implemented inequality is

	\[
		I = \frac{ \mid y_i - y_j \mid}{y_i + y_j}
	\]

\end{frame}

\begin{frame}\frametitle{Fairness views}

	\[
		U_{\text {spectator }}=- \left(y_i-m_i \right)^2
	\]

	\vfill

	\textbf{Egalitarian}: the total earnings are divided equally between the two individuals, \( m_i=\frac{1}{2} \cdot X \).

	\vfill

	\textbf{Meritocratic}: the total earnings are divided proportional to performance, \( m_i=\frac{p_i}{p_i+p_j} \cdot X \).

	\vfill

	\textbf{Libertarian}: the individuals receive their earnings, \( m_i=x_i \).
\end{frame}


\begin{frame}\frametitle{Uncertainty}

	Under limited information, the spectator has to form beliefs:

	\vfill

	\[
		E U_{\text {spectator }}=-E\left(y_i-m_i\right)^2
	\]

	\vfill

	His optimal choice is \( E\left( m_i \right) \).

	\vfill

	\textbf{Performance-ranking uncertainty}: Given a signal \( x_i, x_j \), the spectator's posterior beliefs reflect performance-ranking uncertainty if and only if both \( p_i > p_j \) and \( p_j > p_i \) are in their support.
\end{frame}

\begin{frame}\frametitle{Result: Theory}

	\textbf{Egalitarians}: they always divide equally.

	\vfill

	\textbf{Libertarians}: they always give the earnings to the individuals.

	\vfill

	\textbf{Meritocrats}:

	\vfill

	\begin{prop}
		A Bayesian meritocratic spectator implements in expectation
		the same level of inequality with limited information and full information
		if limited information does not cause \textbf{performance-ranking uncertainty}, and strictly less inequality with limited information than with full information if limited information causes performance-ranking uncertainty.
	\end{prop}

\end{frame}

\begin{frame}\frametitle{Non-Bayesian updating}

	\textbf{Signal-neglecter}: posterior beliefs are equal to the prior.

	\vfill

	The same result on performance ranking uncertainty holds.

	\vfill

	\textbf{Base-rate neglecter}: disregard the prior and use the likelihood of the signal.

	\begin{prop}
		A base-rate-neglecting meritocratic spectator implements in
		expectation strictly more inequality with limited information than with full
		information under some assumptions.
	\end{prop}

\end{frame}

\begin{frame}\frametitle{Experiment}

	Under the assumptions, posterior can be written as follows:

	\[ E \left( p_i \mid x_i \right)= \left( 1-\rho_B \right) \cdot \mu_p+\rho_B \cdot x_i
	\]

	\vfill

	Bayesian updating: \(\rho_B = 0.56\)

	\vfill

	Signal neglecting: \(\rho_B = 0\)

	\vfill

	Base-rate neglecting: \(\rho_B = 1\)
\end{frame}

\begin{frame}\frametitle{Experiment: meritocrats prediction}
	\begin{figure}
		\includegraphics[scale=0.65]{Others/Figures/cap1.png}
	\end{figure}
\end{frame}

\begin{frame}\frametitle{Experiment: implemented inequality}
	\begin{figure}
		\includegraphics[scale=0.58]{Others/Figures/cap.png}
	\end{figure}
\end{frame}


\begin{frame}\frametitle{Experiment: elicited and estimated beliefs}
	\begin{figure}
		\includegraphics[scale=0.65]{Others/Figures/cap2.png}
	\end{figure}
\end{frame}

\begin{frame}\frametitle{Experiment: structural model vs data}
	\begin{figure}
		\includegraphics[scale=0.6]{Others/Figures/cap3.png}
	\end{figure}
\end{frame}

\begin{frame}\frametitle{Experiment: structural analysis}
	\begin{figure}
		\includegraphics[scale=0.85]{Others/Figures/cap4.png}
	\end{figure}
\end{frame}

\begin{frame}\frametitle{Discussion}
	\textbf{Language}: Luck egalitarianism vs Meritocracy.

	\vfill

	\textbf{Underlying motivation}: Why individuals distort information as they do?

	\vfill

	\textbf{Method}: Impartial spectator.
\end{frame}

\begin{comment}

\begin{frame}[noframenumbering,plain]

	\frametitle{References}

	%\nocite{*}
	\bibliography{Others/bib}
	\bibliographystyle{apacite}


\end{frame}

\end{comment}

\end{document}