
\documentclass[usenames,dvipsnames,aspectratio=169,11pt,envcountsect, handout]{beamer}
%handout,
%aspectratio=43

\usepackage{Others/prespreamble}
\usepackage{natbib}
\linespread{1}

% small bibliography
\let\oldthebibliography=\thebibliography
\renewcommand{\thebibliography}[1]{
    \oldthebibliography{#1}
    \setlength{\itemsep}{7pt}
    \small
}

\begin{document}

\section{Intro}

\begin{frame}[noframenumbering,plain]
	\maketitle
	%\begin{figure}[H]
	%\centering
	%\includegraphics[scale=0.4]{Other/Figures/erc.png}
	%\end{figure}

	%\footnotesize{Funding from the European Research Council (ERC) under the European Unions Horizon 2020 research and innovation programme (grant agreement No. 789111 - ERC EvolvingEconomics) is gratefully acknowledged}
\end{frame}
%\frame{\titlepage}

\begin{frame}\frametitle{Do Beliefs exist?}


	"Some scholars we met seemed skeptical to the idea of belief-dependent motivation and in the sequel to the idea of eliciting beliefs \dots


	\vfill


	These would be people who revere “revealed preference,” who argue that beliefs are not real, or at least not observable. Beliefs are merely a theory feature, something that should be viewed only as part of a preference representation \dots

	\vfill

	In our view, this position has little merit." \citep{battigalliBeliefdependentMotivationsPsychological2022}


	\vfill

	Reconcile the two views: agents behave \textbf{as if} they were motivated by beliefs.

\end{frame}


\begin{frame}\frametitle{Literature}

	In the literature of  beliefs-dependent motivations (\textit{BDM}), \textbf{objective beliefs} are present in various forms.

	\vfill

	\begin{wideitemize}
		\item \textit{Psychological games}: system of conditional beliefs \citep{battigalliBeliefdependentMotivationsPsychological2022}.
		\item \textit{Psychological expected utility}: lotteries over future prizes \citep{caplinPsychologicalExpectedUtility2001}.
	\end{wideitemize}

	\vfill

	These models make revealed preference preachers angry.

\end{frame}


\begin{frame}\frametitle{This Presentation}

	I sketch the contours of a theory of \textit{BDM} in a Subjective Expected Utility (\textit{SEU}) framework \citep{savageFoundationsStatistics1972}.

	\vfill

	A model of \textit{BDM} should feature the followings:

	\vfill

	\begin{wideitemize}
		\item Beliefs are \textcolor{blue}{produced} by the agent;
		\item \textcolor{blue}{Trade-off} between accuracy and wishful thinking;
		\item Beliefs \textcolor{blue}{depend on material reward}. (\textcolor{red}{Problematic!})
	\end{wideitemize}

	\vfill

	I showcase these features in examples and attempt to account for them.


\end{frame}

\begin{frame}{Example: guilt aversion}
	I am a cab driver, and I have expectations regarding the tip I receive from a client.

	\vfill

	The client suffers guilt if he does not match my expectations.

	\vfill

	It would be very convenient for me to have high expectations.

	\vfill

	If the client has $10\$$, I believe he will tip $10\$$.

	\vfill

	But if he has $50\$$, I believe he will tip $50\$$.


\end{frame}

\begin{frame}{Example: a bet}

	\begin{minipage}{0.30\textwidth}
		\[
			\begin{array}{c|cc}
				           & p & 1 - p \\
				           & H & L     \\
				\hline a_H & 1 & 0     \\
				a_L        & 0 & 1 + b
			\end{array}
		\]
	\end{minipage}
	\begin{minipage}{0.60\textwidth}
		A \textit{BDM} agent's preferences increase in $p$.
		\vfill
		\center \textcolor{blue}{(Production)}
	\end{minipage}

	\vfill

	Without \textit{BDM}, $a_H$ is preferred to $a_L$ when $p \geq \frac{1 + b}{2 + b}$.

	\vfill

	When $p \geq \frac{1 + b}{2 + b}$, there is no "cost" of increasing $p$, it is "free" to set $p = 1$. \textcolor{blue}{(Trade-off)}

	\vfill

	For different $b$, the agent will have different $p$. \textcolor{blue}{(Dependence on material outcome)}

	\vfill

	What does it mean to "elicit beliefs"?

\end{frame}

\begin{frame}{Savage Model}

	A \textbf{finite} decision problem comprises:

	\vfill

	\begin{wideitemize}
		\item Set of \textbf{uncertain states} $S$;
		\item Set of \textbf{outcomes} $X$;
		\item \textbf{Acts} mapping uncertain states to outcomes $a: S \rightarrow X$;
		\item \textbf{Preferences} $\succsim$ on the set of acts $A$, represented by
	\end{wideitemize}

	\begin{equation}
		U(a) = \sum_{s \in S} u(a(s)) \cdot p(s) \: .
		\tag{SEU}
	\end{equation}

\end{frame}

\begin{frame}{Probability dependent outcome}

	Define the \textbf{outcome valuation at} $p$ as $t \left( \cdot, p \right) : X \rightarrow X$ for each belief $p$.

	\vfill

	Preferences are now represented by

	\vfill

	\begin{equation}
		U(a) = \sum_{s \in S} u \left( \textcolor{blue}{t \left(a\left(s\right),p \right)} \right) \cdot p(s) \:
		\tag{PDOSEU}
	\end{equation}

	\begin{equation}
		U(a) = \sum_{s \in S} u(\textcolor{blue}{a(s)}) \cdot p(s) \: .
		\tag{SEU}
	\end{equation}

	\vfill

	This model is inspired by \cite{karniUtilityTheoryProbabilitydependent1992}, set in an objective probability framework.

\end{frame}

\begin{frame}{Previous Example}

	\begin{minipage}{0.45\textwidth}
		\[
			\begin{array}{c|cc}
				           & p & 1 - p \\
				           & H & L     \\
				\hline a_H & 1 & 0     \\
				a_L        & 0 & 1 + b
			\end{array}
		\]
	\end{minipage}
	\begin{minipage}{0.45\textwidth}
		Set $t \left( x , p \right) = x + \mathbf{1} \left\{p \geq \frac{1+b}{2+b} \right\}$.
	\end{minipage}

	\vfill

	If $p \geq \frac{1+b}{2+b}$, the outcome evaluation is "kicked" without a change in beliefs.

	\vfill

	The quantity $x + \mathbf{1} \left\{p \geq \frac{1+b}{2+b} \right\}$ represents a "compensation" for not distorting beliefs.

	\vfill

	Beliefs are not affected by a change in $b$.


\end{frame}

\section{Conclusion}

\begin{frame}{Conclusion}

	Current modeling of belief-dependent motivations relies on objective beliefs.

	\vfill

	Attempt to capture belief-dependent motivations in subjective expected utility.

	\vfill

	Hard to disentangle probabilities affecting conjectures on uncertain states and probabilities involving outcome evaluations.

\end{frame}

\begin{frame}[noframenumbering,plain]

	\frametitle{References}

	%\nocite{*}
	\bibliography{Others/bib}
	\bibliographystyle{apacite}


\end{frame}


\end{document}