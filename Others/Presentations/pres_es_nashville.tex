
\documentclass[usenames,dvipsnames,aspectratio=169,11pt, envcountsect]{beamer}
%handout,
%aspectratio=43

\usepackage{Others/prespreamble}
\usepackage{natbib}
\definecolor{bleudefrance}{rgb}{0.19, 0.55, 0.91}
\linespread{1}
\usepackage{mathrsfs}

% small bibliography
\let\oldthebibliography=\thebibliography
\renewcommand{\thebibliography}[1]{
    \oldthebibliography{#1}
    \setlength{\itemsep}{2pt}
    \tiny
}

\begin{document}

\section{Introduction}

\begin{frame}[noframenumbering,plain]
	\maketitle
	%\begin{figure}[H]
	%\centering
	%\includegraphics[scale=0.4]{Other/Figures/erc.png}
	%\end{figure}

	%\footnotesize{Funding from the European Research Council (ERC) under the European Unions Horizon 2020 research and innovation programme (grant agreement No. 789111 - ERC EvolvingEconomics) is gratefully acknowledged}
\end{frame}
%\frame{\titlepage}

\begin{frame}\frametitle{Motivation}

	Investors overreact to information, consumers avoid learning about firms' unethical practices, and patients at health risk do not learn about their condition.

	\vfill

	Theories of belief-dependent motivations (BDM) explain these phenomena.

	\vfill \pause

	Three drawbacks:

	\vfill

	\begin{wideitemize}
		\item lack of preferences and beliefs identification;
		\item impossibility to distinguish "desired" from "undesired" beliefs;
		\item unknown relation between preferences and belief revision.
	\end{wideitemize}

	\vfill

	\textbf{Question}: can we develop a testable theory of BDM?

\end{frame}

\begin{frame}\frametitle{This Paper}

	I develop a theory of BDM in a dynamic setting encompassing previous model.

	\vfill

	Frist, The individual chooses both a menu and information.

	\vfill

	Choosing from the menu, she alters her Bayesian update to satisfy her preferences.

	\vfill

	Axiomatic analysis identifies preferences, prior beliefs and distorted posteriors.

	\vfill

	\textbf{Main result}: representation of BDM preferences and belief updating rules.

\end{frame}

\begin{frame}{Illustrative Example}

	An investor decides whether to check the status of her portfolio.

	\vfill

	After checking, she decides whether to invest more \( \left( i \right) \) or withdraw any feasible amount of money, which could be high \( \left( \overline{w} \right) \) or low \( \left( w \right) \).

	\vfill

	\begin{table}[H]
		\centering
		\begin{minipage}{0.29\textwidth}

		\end{minipage}\hspace{0.3cm} % Adjust the horizontal space between the tables and the symbol
		% Symbol goes here
		\begin{minipage}{0.29\textwidth}
			\centering
			\begin{tabular}{c | c}
				\multicolumn{2}{c}{\textbf{Check}}                                                                            \\
				State                        & Actions                                                                        \\
				\hline
				{\color{bleudefrance}Good}   & \multirow{2}{*}{{\color{bleudefrance}\( i, \left[ 0, \overline{w} \right] \)}} \\
				{\color{bleudefrance}Normal} &                                                                                \\
				Bad                          & \(  i, \left[0, w \right] \)                                                   \\
			\end{tabular}
			\vspace{0.5cm} % Adjust vertical space between tables and caption
		\end{minipage}\hspace{0.3cm} % Adjust the horizontal space between the tables and the symbol
		% Symbol goes here
		\begin{minipage}{0.29\textwidth}

		\end{minipage}
		%\caption{Commitment under positive prior belief to avoid excessive investment.} % Add your caption here
		%\label{tab:commitment}
	\end{table} \pause

	\vfill

	Upon observing a high amount in it she infers the status of the market is not bad.

	\vfill

	When she sees a low amount, she knows the status of the market is bad.

\end{frame}

\begin{frame}[noframenumbering]{Illustrative Example}

	She can't make any inferences or do anything if she does not check.

	\vfill

	\begin{table}[H]
		\centering
		\begin{minipage}{0.29\textwidth}

		\end{minipage}\hspace{0.3cm} % Adjust the horizontal space between the tables and the symbol
		% Symbol goes here
		\begin{minipage}{0.29\textwidth}
			\centering
			\begin{tabular}{c | c}
				\multicolumn{2}{c}{\textbf{Check}}                                                                            \\
				State                        & Actions                                                                        \\
				\hline
				{\color{bleudefrance}Good}   & \multirow{2}{*}{{\color{bleudefrance}\( i, \left[ 0, \overline{w} \right] \)}} \\
				{\color{bleudefrance}Normal} &                                                                                \\
				Bad                          & \(  i, \left[0, w \right] \)                                                   \\
			\end{tabular}
			\vspace{0.5cm} % Adjust vertical space between tables and caption
		\end{minipage}\hspace{0.7cm} % Adjust the horizontal space between the tables and the symbol
		% Symbol goes here
		\begin{minipage}{0.29\textwidth}
			\centering
			\begin{tabular}{c | c}
				\multicolumn{2}{c}{\textbf{Not Check}} \\
				State  & Actions                       \\
				\hline
				Good   & \multirow{3}{*}{ \( 0 \)}     \\
				Normal &                               \\
				Bad    &                               \\
			\end{tabular}
			\vspace{0.5cm} % Adjust vertical space between tables and caption
		\end{minipage}
		%\caption{Commitment under positive prior belief to avoid excessive investment.} % Add your caption here
		%\label{tab:commitment}
	\end{table}

\end{frame}

\begin{frame}[noframenumbering]{Illustrative Example}

	She could also check and committ not to invest, by delegating to a financial advisor.

	\vfill

	\begin{table}[H]
		\centering
		\begin{minipage}{0.29\textwidth}
			\centering
			\begin{tabular}{c | c}
				\multicolumn{2}{c}{\textbf{Delegate}}                                                                     \\
				State                        & Actions                                                                    \\
				\hline
				{\color{bleudefrance}Good}   & \multirow{2}{*}{{\color{bleudefrance}\( \left[0, \overline{w} \right] \)}} \\
				{\color{bleudefrance}Normal} &                                                                            \\
				Bad                          & \( \left[0, w \right]\)                                                    \\
			\end{tabular}
			\vspace{0.5cm} % Adjust vertical space between tables and caption
		\end{minipage}\hspace{0.5cm} % Adjust the horizontal space between the tables and the symbol
		% Symbol goes here
		\begin{minipage}{0.29\textwidth}
			\centering
			\begin{tabular}{c | c}
				\multicolumn{2}{c}{\textbf{Check}}                                                                            \\
				State                        & Actions                                                                        \\
				\hline
				{\color{bleudefrance}Good}   & \multirow{2}{*}{{\color{bleudefrance}\( i, \left[ 0, \overline{w} \right] \)}} \\
				{\color{bleudefrance}Normal} &                                                                                \\
				Bad                          & \(  i, \left[0, w \right] \)                                                   \\
			\end{tabular}
			\vspace{0.5cm} % Adjust vertical space between tables and caption
		\end{minipage}\hspace{0.5cm} % Adjust the horizontal space between the tables and the symbol
		% Symbol goes here
		\begin{minipage}{0.29\textwidth}
			\centering
			\begin{tabular}{c | c}
				\multicolumn{2}{c}{\textbf{Not Check}} \\
				State  & Actions                       \\
				\hline
				Good   & \multirow{3}{*}{ \( 0 \)}     \\
				Normal &                               \\
				Bad    &                               \\
			\end{tabular}
			\vspace{0.5cm} % Adjust vertical space between tables and caption
		\end{minipage}
		%\caption{Commitment under positive prior belief to avoid excessive investment.} % Add your caption here
		%\label{tab:commitment}
	\end{table}

\end{frame}

\begin{frame}[noframenumbering]{Illustrative Example}

	She anticipates to overweight evidence and invest too much.

	\vfill

	Therefore, she prefers to commit, but also wants to obtain information.

	\vfill

	\begin{table}[H]
		\centering
		\begin{minipage}{0.29\textwidth}
			\centering
			\begin{tabular}{c | c}
				\multicolumn{2}{c}{\textbf{Delegate}}                                                                     \\
				State                        & Actions                                                                    \\
				\hline
				{\color{bleudefrance}Good}   & \multirow{2}{*}{{\color{bleudefrance}\( \left[0, \overline{w} \right] \)}} \\
				{\color{bleudefrance}Normal} &                                                                            \\
				Bad                          & \( \left[0, w \right]\)                                                    \\
			\end{tabular}
			\vspace{0.5cm} % Adjust vertical space between tables and caption
		\end{minipage}\hspace{0.3cm} % Adjust the horizontal space between the tables and the symbol
		\( \succ \) % Symbol goes here
		\begin{minipage}{0.29\textwidth}
			\centering
			\begin{tabular}{c | c}
				\multicolumn{2}{c}{\textbf{Check}}                                                                            \\
				State                        & Actions                                                                        \\
				\hline
				{\color{bleudefrance}Good}   & \multirow{2}{*}{{\color{bleudefrance}\( i, \left[ 0, \overline{w} \right] \)}} \\
				{\color{bleudefrance}Normal} &                                                                                \\
				Bad                          & \(  i, \left[0, w \right] \)                                                   \\
			\end{tabular}
			\vspace{0.5cm} % Adjust vertical space between tables and caption
		\end{minipage}\hspace{0.3cm} % Adjust the horizontal space between the tables and the symbol
		\( \succ \) % Symbol goes here
		\begin{minipage}{0.29\textwidth}
			\centering
			\begin{tabular}{c | c}
				\multicolumn{2}{c}{\textbf{Not Check}} \\
				State  & Actions                       \\
				\hline
				Good   & \multirow{3}{*}{ \( 0 \)}     \\
				Normal &                               \\
				Bad    &                               \\
			\end{tabular}
			\vspace{0.5cm} % Adjust vertical space between tables and caption
		\end{minipage}
		\caption{Commitment under positive prior belief to avoid excessive investment.} % Add your caption here
		\label{tab:commitment}
	\end{table}

	\vfill

	"Cognitive" non-Bayesian updating \citep{epsteinAxiomaticModelNonBayesian2006} cannot rationalise this behaviour.

\end{frame}

\begin{frame}[noframenumbering]{Illustrative Example}

	If the investor expects the status of the market to be bad, she prefers not to check the portfolio at all to avoid receiving unpleasant information.

	\vfill

	\begin{table}[H]
		\centering
		\begin{minipage}{0.29\textwidth}
			\centering
			\begin{tabular}{c | c}
				\multicolumn{2}{c}{\textbf{Not Check}} \\
				State  & Actions                       \\
				\hline
				Good   & \multirow{3}{*}{ \( 0 \)}     \\
				Normal &                               \\
				Bad    &                               \\
			\end{tabular}
			\vspace{0.5cm} % Adjust vertical space between tables and caption
		\end{minipage}\hspace{0.3cm} % Adjust the horizontal space between the tables and the symbol
		\( \succ \) % Symbol goes here
		\begin{minipage}{0.29\textwidth}
			\centering
			\begin{tabular}{c | c}
				\multicolumn{2}{c}{\textbf{Delegate}}                                                                     \\
				State                        & Actions                                                                    \\
				\hline
				{\color{bleudefrance}Good}   & \multirow{2}{*}{{\color{bleudefrance}\( \left[0, \overline{w} \right] \)}} \\
				{\color{bleudefrance}Normal} &                                                                            \\
				Bad                          & \( \left[0, w \right]\)                                                    \\
			\end{tabular}
			\vspace{0.5cm} % Adjust vertical space between tables and caption
		\end{minipage}\hspace{0.3cm} % Adjust the horizontal space between the tables and the symbol
		\( \succ \) % Symbol goes here
		\begin{minipage}{0.29\textwidth}
			\centering
			\begin{tabular}{c | c}
				\multicolumn{2}{c}{\textbf{Check}}                                                                            \\
				State                        & Actions                                                                        \\
				\hline
				{\color{bleudefrance}Good}   & \multirow{2}{*}{{\color{bleudefrance}\( i, \left[ 0, \overline{w} \right] \)}} \\
				{\color{bleudefrance}Normal} &                                                                                \\
				Bad                          & \(  i, \left[0, w \right] \)                                                   \\
			\end{tabular}
			\vspace{0.5cm} % Adjust vertical space between tables and caption
		\end{minipage}
		\caption{Information avoidance under negative prior beliefs, "ostrich effect".} % Add your caption here
		\label{tab:oistrich}
	\end{table}

	Both excessive trading and the ostrich effect constitutes empirical puzzles in finance \citep{danielOverconfidentInvestorsPredictable2015,golmanInformationAvoidance2017}.

\end{frame}

\begin{frame}\frametitle{Literature}

	\begin{wideitemize}
		\item \textit{Decision Theory.} \cite{liangInformationdependentExpectedUtility2017}, \cite{dillenbergerAdditivebeliefbasedPreferences2020} \cite{rommeswinkelPreferenceKnowledge2023}.

		\vspace{0.3cm}
		\underline{Contribution}: \textbf{Belief revision rule}.
		\item \textit{Menu Choice.} \cite{gulTemptationSelfControl2001}, \cite{ozdenorenCompletingStateSpace2002}, \cite{epsteinAxiomaticModelNonBayesian2006}, \cite{epsteinColdFeet2007}.

		\vspace{0.3cm}
		\underline{Contribution}: \textbf{Novel primitive object of choice}.
		\item \textit{Belief-Dependent Motivations.} \cite{eliazCanAnticipatoryFeelings2006}, \cite{benabou2016mindful}, \cite{golmanInformationAvoidance2017}, \cite{battigalliBeliefdependentMotivationsPsychological2022}.

		\vspace{0.3cm}
		\underline{Contribution}: \textbf{Interaction between preferences and belief revision}.
	\end{wideitemize}

\end{frame}

\section{Body}

\begin{frame}\frametitle{Model}
	\begin{itemize}
		\item
		      compact metric outcome set \(X\), the set of lotteries \(\Delta \left( X \right)\) is compact metric under the weak convergence topology; \pause
		\item
		      finite state spaces \(S\); \pause
		\item
		      the set of Anscombe-Aumann (AA) acts over \(S\) is \(\Delta \left( X \right)^{S}\), a generic act is \(f: S \longrightarrow\) \(\Delta(X)\); \pause
		\item
		      a closed subset \(M\) of \(\Delta \left( X \right)^{S}\) is a menu of acts over \(S\); \pause
		\item
		      the set of menus is \(\mathcal{M}\), it is compact metric under the Hausdorff metric, the set of lotteries with finite support over it is \(\Delta^{0} \left( \mathcal{M} \right)\); \pause
		\item
		      a contingent menu is \(F: S \rightarrow \Delta^{0} \left( \mathcal{M} \right) \), the probability that menu \( M \) realises if \(s\) is the true state is \(F_{s} \left( M \right) \); \pause
		\item
		      the set of all contingent menus is \(\mathcal{C}= \Delta^{0} \left( \mathcal{M} \right)^S \); \pause
		\item
		      time \(0\) preference \(\succsim\) is defined on \( \mathcal{C} \).
	\end{itemize}
\end{frame}

\begin{frame}{Information}
	The likelihood of state \(s\) after realisation of menu \(M\) from the contingent menu \(F\) is

	\vfill

	\[ \ell_{M, F} \left( s \right) : = \frac{F_{s} \left( M \right)}{ \sum_{s^{\prime} \in S} F_{s^{\prime}} \left( M \right)} .
	\]

	\vfill

	Given any contingent menu \(F\) and menu \(M\), the vector of likelihoods is \(\ell_{M, F}\).
\end{frame}

\begin{frame}{\cite{gulTemptationSelfControl2001}}

	In the temptation and self-control model, behavior is represented by the following

	\vfill

	\[
		\mathcal{U} \left( M \right) = \max_{f \in M} \left\{ U \left( f \right) + V \left( f \right) - \max_{f^{\prime} \in M} V \left( f^{\prime} \right) \right\} . \pause
	\]

	\vfill

	Cost of self-control is

	\vfill

	\[
		\max_{f^{\prime} \in M} V \left( f^{\prime} \right) - V \left( f \right) .
	\]

\end{frame}

\begin{frame}{Utility}

	Individual's behavior in this paper is represented by the following model

	\vfill

	\begin{equation}\label{eq:contmenu1}
		\mathscr{U}(F)= \sum_{M} \left( \sum_{s} F_{s} \left( M \right) \right) \mathcal{U} \left(M ; \ell_{M, F} \right) \: ;
	\end{equation} \pause

	\vfill

	\begin{equation}\label{eq:menu1}
		\begin{aligned}
			\mathcal{U} \left(M ; \ell_{M, F} \right) = & \max _{f \in M}\left\{\sum_{s} u \left( f_{s} ; \ell_{M, F} \right) p_{\ell_{M, F}} \left( s \right) \pause +\alpha _{\ell_{M, F}} \sum_{s} u \left( f_{s} ; \ell^{*}_{M, F} \right) p_{\ell^{*}_{M, F}} \left( s \right) \right\} \\ \pause
			                                            & -\max _{f^{\prime} \in M} \alpha _{\ell_{M, F}} \sum_{s} u\left(f^{\prime}_{s} ; \ell^{*}_{M, F} \right) p_{\ell^{*}_{M, F}} \left( s \right) \: .
		\end{aligned}
	\end{equation}

	\vfill \pause

	Belief-dependent motivations \textbf{imply} non-Bayesian updating.

\end{frame}

\begin{frame}{Interpretation}

	Choice at period \( 2 \) is described by the following

	\vfill


	\[
		\max _{f \in M} \left[ \sum_{s} u \left( f_{s} ; \ell_{M, F} \right) p_{\ell_{M, F}} \left( s \right) +\alpha _{\ell_{M, F}} \sum_{s} u \left( f_{s} ; \ell^{*}_{M, F} \right) p_{\ell^{*}_{M, F}} \left( s \right) \right].
	\]

	\vfill \pause


	When choosing act \( f \) from menu \( M \) after realisation of the likelihood \( \ell_{M, F} \), the utility cost of temptation is

	\vfill

	\[
		\alpha _{\ell_{M, F}} \left[ \max _{f^{\prime} \in M} \sum_{s} u\left(f^{\prime}_{s} ; \ell^{*}_{M, F} \right) p_{\ell^{*}_{M, F}} \left( s \right) - \sum_{s} u \left( f_{s} ; \ell^{*}_{M, F} \right) p_{\ell^{*}_{M, F}} \left( s \right) \right] .
	\]

\end{frame}

\begin{comment}

\begin{frame}{Axioms}


	\begin{axiom}
	\end{axiom}

	\vfill

	(\textbf{Order}). The ranking \(\succsim\) is complete and transitive.

	\vfill

	(\textbf{Continuity}). For all contingent menus \( F^{\prime} \) the sets \( \left\{ F \mid F \succsim F^{\prime} \right\} \) and \( \left\{ F \mid F \precsim F^{\prime} \right\} \) are closed.

	\vfill

	(\textbf{Nondegeneracy}). There exist \(y, y^{\prime}\) in \(X\) for which \(y \succ y^{\prime}\).

	\vfill

	(\textbf{\textbf{Full Support}}). For each state \(s\), there exist contingent menus \(F \) and \(F^{\prime}\) such that \(F \nsim F^{\prime}\), where for all menus \( M \) it holds that \(F\left( s^{\prime}\right) \left( M \right) =F^{\prime}\left( s^{\prime}\right) \left( M \right)\) for every \(s^{\prime} \neq s \).

\end{frame}

\begin{frame}{Identical Inference Independence}

	\begin{df}\label{def:ii}
		(\textbf{Identical Inference (II)})
		Two contingent menus \(F\) and \(F^{\prime}\) satisfy \textbf{identical inference} if, for each \(M \in supp \left(F \right) \cap supp \left( F^{\prime} \right)\) their likelihood is the same \(\ell \left( M_F \right) = \ell\left( M_{F^{\prime}} \right)\).
	\end{df}

	\vfill

	\begin{table}[H]
		\centering
		\begin{minipage}{0.45\textwidth}
			\centering
			\begin{tabular}{c | c}
				State                          & Actions                                                                                       \\
				\hline
				{\color{bleudefrance} High}    & {\color{bleudefrance}\( h \left\{ 5 \right\} \) } \(+ \left(1-h \right) \left[5,10 \right] \) \\
				{\color{bleudefrance} Medium } & {\color{bleudefrance}\( m \left\{ 5 \right\} \) } \(+ \left(1-m \right) \left[0,5 \right] \)  \\
				Low                            & \( \left[0,3 \right] \)                                                                       \\
			\end{tabular}
			\vspace{0.5cm} % Adjust vertical space between tables and caption
		\end{minipage}\hspace{0.5cm} % Adjust the horizontal space between the tables and the symbol
		\hspace{0.25cm}
		\begin{minipage}{0.45\textwidth}
			\centering
			\begin{tabular}{c | c}
				State                          & Actions                                                                                         \\
				\hline
				{\color{bleudefrance} High}    & {\color{bleudefrance}\( h \left\{ 3,5 \right\} \) } \(+ \left(1-h \right) \left[5,10 \right] \) \\
				{\color{bleudefrance} Medium } & {\color{bleudefrance}\( m \left\{ 3,5 \right\} \) } \(+ \left(1-m \right) \left[0,5 \right] \)  \\
				Low                            & \( \left[0,3 \right] \)                                                                         \\
			\end{tabular}
			\vspace{0.5cm} % Adjust vertical space between tables and caption
		\end{minipage}
		%\caption{Set-Betweenness} % Add your caption here
		%\label{tab:sbetweenness}
	\end{table} \pause

	\begin{axiom}\label{ax:independence}
		(\textbf{II Independence}). For all \(0<\lambda \leq 1\) and contingent menus \(F, F^{\prime}, F^{\prime \prime} \) such that \(F\) and \(F^{\prime \prime}\) satisfy II and \(F^{\prime}\) and \(F^{\prime \prime}\) satisfy II, \(F \succsim F^{\prime}\) if and only if \(\lambda F+ \left( 1-\lambda \right) F^{\prime \prime} \succsim \lambda F^{\prime} + \left( 1-\lambda \right) F^{\prime \prime}\).
	\end{axiom}

\end{frame}

\begin{frame}{Set Betweenness}
	Any contingent menu \( F \), for any menu \( M \), can be denoted with \( \left( F_{-M}, M \right) \).

	\vfill

	The two contingent menus \( \left( F_{-M}, M \right) \) and \( \left( F_{-M^{\prime}}, M^{\prime} \right) \) are equivalent except the realisation \( M \) of the first is substituted with \( M^{\prime} \) in the second. \pause

	\vfill

	\begin{axiom}\label{ax:betweenness}
		(\textbf{Set-Betweenness}). For all contingent menus \( \left( F_{-M}, M \right) \) and \( \left( F_{-M^{\prime}}, M^{\prime} \right) \),

		\[
			\left( F_{-M}, M \right)  \succsim \left( F_{-M^{\prime}}, M^{\prime} \right) \Rightarrow \left( F_{-M}, M \right)  \succsim \left( F_{-M \cup M^{\prime}}, M \cup M^{\prime} \right) \succsim \left( F_{-M^{\prime}}, M^{\prime} \right) .
		\]

	\end{axiom}
\end{frame}

\begin{frame}{Set Betweenness illustrated}
	\[
		\left( F_{-M}, M \right)  \succsim \left( F_{-M^{\prime}}, M^{\prime} \right) \Rightarrow \left( F_{-M}, M \right)  \succsim \left( F_{-M \cup M^{\prime}}, M \cup M^{\prime} \right) \succsim \left( F_{-M^{\prime}}, M^{\prime} \right)
	\]

	\vfill

	\begin{table}[H]
		\centering
		\begin{minipage}{0.45\textwidth}
			\centering
			\begin{tabular}{c | c}
				State                          & Actions                                                                                       \\
				\hline
				{\color{bleudefrance} High}    & {\color{bleudefrance}\( h \left\{ 5 \right\} \) } \(+ \left(1-h \right) \left[5,10 \right] \) \\
				{\color{bleudefrance} Medium } & {\color{bleudefrance}\( m \left\{ 5 \right\} \) } \(+ \left(1-m \right) \left[0,5 \right] \)  \\
				Low                            & \( \left[0,3 \right] \)                                                                       \\
			\end{tabular}
			%\vspace{0.5cm} % Adjust vertical space between tables and caption
		\end{minipage}\hspace{0.25cm} % Adjust the horizontal space between the tables and the symbol
		\( \succsim \) % Symbol goes here
		\hspace{0.25cm}
		\begin{minipage}{0.45\textwidth}
			\centering
			\begin{tabular}{c | c}
				State                          & Actions                                                                                       \\
				\hline
				{\color{bleudefrance} High}    & {\color{bleudefrance}\( h \left\{ 3 \right\} \) } \(+ \left(1-h \right) \left[5,10 \right] \) \\
				{\color{bleudefrance} Medium } & {\color{bleudefrance}\( m \left\{ 3 \right\} \) } \(+ \left(1-m \right) \left[0,5 \right] \)  \\
				Low                            & \( \left[0,3 \right] \)                                                                       \\
			\end{tabular}
			%\vspace{0.5cm} % Adjust vertical space between tables and caption
		\end{minipage}
		%\caption{Set-Betweenness} % Add your caption here
		%\label{tab:sbetweenness}
	\end{table}
	\vspace{-0.5cm}
	\begin{center}
		\( \Downarrow \)
	\end{center}
	\vspace{-0.5cm}
	\begin{table}[H]
		\centering
		\begin{minipage}{0.45\textwidth}
			\centering
			\begin{tabular}{c | c}
				State                          & Actions                                                                                       \\
				\hline
				{\color{bleudefrance} High}    & {\color{bleudefrance}\( h \left\{ 5 \right\} \) } \(+ \left(1-h \right) \left[5,10 \right] \) \\
				{\color{bleudefrance} Medium } & {\color{bleudefrance}\( m \left\{ 5 \right\} \) } \(+ \left(1-m \right) \left[0,5 \right] \)  \\
				Low                            & \( \left[0,3 \right] \)                                                                       \\
			\end{tabular}
			%\vspace{0.5cm} % Adjust vertical space between tables and caption
		\end{minipage}\hspace{0.25cm} % Adjust the horizontal space between the tables and the symbol
		\( \succsim \) % Symbol goes here
		\hspace{0.25cm}
		\begin{minipage}{0.45\textwidth}
			\centering
			\begin{tabular}{c | c}
				State                          & Actions                                                                                         \\
				\hline
				{\color{bleudefrance} High}    & {\color{bleudefrance}\( h \left\{ 3,5 \right\} \) } \(+ \left(1-h \right) \left[5,10 \right] \) \\
				{\color{bleudefrance} Medium } & {\color{bleudefrance}\( m \left\{ 3,5 \right\} \) } \(+ \left(1-m \right) \left[0,5 \right] \)  \\
				Low                            & \( \left[0,3 \right] \)                                                                         \\
			\end{tabular}
			%\vspace{0.5cm} % Adjust vertical space between tables and caption
		\end{minipage}
		%\caption{Set-Betweenness} % Add your caption here
		%\label{tab:sbetweenness}
	\end{table}
\end{frame}

\begin{frame}{Preferred Likelihood}
	The set of likelihoods giving positive weights only to states in event \( E \) is \( \mathcal{L}_E \).

	\vfill

	The notation \( \left( F_{-M}, M_{\ell} \right) \) means \( M \) induces likelihood \( \ell \). For each event \( E \)

	\vfill

	\[
		\mathcal{L}_{E}^{*} := \left\{ \ell \in \mathcal{L}_{E} \: \mid \: \left( F_{-\Delta \left(X \right)}, \Delta \left(X \right)_{\ell} \right) \succsim \left( F_{-\Delta \left(X \right)}, \Delta \left(X \right)_{\ell^{\prime}} \right) \: \text{for all} \: \ell^{\prime} \in \mathcal{L}_E \right\} .
	\] \pause

	\vfill

	For each menu \( M \) and likelihood \( \ell_{E} \)

	\vfill

	\[
		M^{*}_{\ell_{E}} := \left\{ f \in M \: \mid \: \left( F_{-\left\{ f \right\}}, \left\{ f \right\}_{\ell_{E}} \right) \succsim \left( F_{-\left\{ f^{\prime} \right\}}, \left\{ f^{\prime} \right\}_{\ell_{E}} \right) \: \text{for all} \: f^{\prime} \in M \right\} .
	\]
\end{frame}


\begin{frame}{Strategic Rationality for Best Likelihood}
	\begin{axiom}\label{ax:srbl}

		(\textbf{Strategic Rationality for Best Likelihood}). For all not empty menus \( M, M^{\prime} \) and likelihoods \( \ell_E \in \mathcal{L}_{E} \), if \( \left( M \cup M^{\prime} \right)^{*}_{\ell_{E}} \cap \left( M \cup M^{\prime} \right)^{*}_{\ell^{*}_{E}} \neq \emptyset \) for some \( \ell_E^{*} \in \mathcal{L}_E^{*} \), then

		\[
			\left( F_{-M}, M_{\ell_{E}} \right) \succsim \left( F_{-M^{\prime}}, M^{\prime}_{\ell_{E}} \right) \Rightarrow \left( F_{-M}, M_{\ell_{E}} \right) \sim \left( F_{-M \cup M^{\prime}}, M \cup M^{\prime}_{\ell_{E}} \right) .
		\]

	\end{axiom} \pause

	\begin{table}[H]
		\centering
		\begin{minipage}{0.45\textwidth}
			\centering
			\begin{tabular}{c | c}
				State                          & Actions                                                                                       \\
				\hline
				{\color{bleudefrance} High}    & {\color{bleudefrance}\( h \left\{ 5 \right\} \) } \(+ \left(1-h \right) \left[5,10 \right] \) \\
				{\color{bleudefrance} Medium } & {\color{bleudefrance}\( m \left\{ 5 \right\} \) } \(+ \left(1-m \right) \left[0,5 \right] \)  \\
				Low                            & \( \left[0,3 \right] \)                                                                       \\
			\end{tabular}
		\end{minipage}\hspace{0.25cm} % Adjust the horizontal space between the tables and the symbol
		\( \succ \) % Symbol goes here
		\hspace{0.25cm}
		\begin{minipage}{0.45\textwidth}
			\centering
			\begin{tabular}{c | c}
				State                          & Actions                                                                                       \\
				\hline
				{\color{bleudefrance} High}    & {\color{bleudefrance}\( h \left\{ 7 \right\} \) } \(+ \left(1-h \right) \left[5,10 \right] \) \\
				{\color{bleudefrance} Medium } & {\color{bleudefrance}\( m \left\{ 7 \right\} \) } \(+ \left(1-m \right) \left[0,5 \right] \)  \\
				Low                            & \( \left[0,3 \right] \)                                                                       \\
			\end{tabular}
		\end{minipage}
		%\caption{Set-Betweenness} % Add your caption here
		%\label{tab:srbl}
	\end{table}
	\vspace{-0.25cm}
	\begin{center}
		\( \Downarrow \)
	\end{center}
	\vspace{-0.25cm}
	\begin{table}[H]
		\centering
		\begin{minipage}{0.45\textwidth}
			\centering
			\begin{tabular}{c | c}
				State                          & Actions                                                                                       \\
				\hline
				{\color{bleudefrance} High}    & {\color{bleudefrance}\( h \left\{ 5 \right\} \) } \(+ \left(1-h \right) \left[5,10 \right] \) \\
				{\color{bleudefrance} Medium } & {\color{bleudefrance}\( m \left\{ 5 \right\} \) } \(+ \left(1-m \right) \left[0,5 \right] \)  \\
				Low                            & \( \left[0,3 \right] \)                                                                       \\
			\end{tabular}
			%\vspace{0.5cm} % Adjust vertical space between tables and caption
		\end{minipage}\hspace{0.25cm} % Adjust the horizontal space between the tables and the symbol
		\( \sim \) % Symbol goes here
		\hspace{0.25cm}
		\begin{minipage}{0.45\textwidth}
			\centering
			\begin{tabular}{c | c}
				State                          & Actions                                                                                         \\
				\hline
				{\color{bleudefrance} High}    & {\color{bleudefrance}\( h \left\{ 5,7 \right\} \) } \(+ \left(1-h \right) \left[5,10 \right] \) \\
				{\color{bleudefrance} Medium } & {\color{bleudefrance}\( m \left\{ 5,7 \right\} \) } \(+ \left(1-m \right) \left[0,5 \right] \)  \\
				Low                            & \( \left[0,3 \right] \)                                                                         \\
			\end{tabular}
			%\vspace{0.5cm} % Adjust vertical space between tables and caption
		\end{minipage}
		%\caption{Strategic Rationality for Best Likelihood} % Add your caption here
		%\label{tab:srbl}
	\end{table}

\end{frame}

\begin{frame}{State Independence}

	Menus of objective lotteries are denoted with \( L \subseteq \Delta \left( X \right) \).

	\vfill

	The notation \( f s f^{\prime} \) indicates an act being \( f \left( s \right) \) in state \( s \) and \( f^{\prime} \left( s^{\prime} \right) \) in all states \( s^{\prime} \neq s \).

	\vfill

	For any state \( s \) and menus \( M, M^{\prime} \), define the menu \( M s M^{\prime} := \left\{ f s f^{\prime} \: \mid \: f \in M, f^{\prime} \in M^{\prime} \right\} \). \pause

	\vfill

	\begin{axiom}\label{ax:sindependence}

		(\textbf{\textbf{State Independence}}). For all states \( s, s^{\prime} \) and menus \( L, L^{\prime}, M \),

		\[
			\left( F_{-L s M}, L s M \right) \succsim \left( F_{-L^{\prime} s M}, L^{\prime} s M \right) \: \Rightarrow \: \left( F_{-L {s^{\prime}} M}, L {s^{\prime}} M \right) \succsim \left( F_{-L^{\prime} {s^{\prime}} M}, L^{\prime} {s^{\prime}} M \right) .
		\]

	\end{axiom}
\end{frame}

\begin{frame}{Result}

	\begin{brotheorem}
	\end{brotheorem}

	\vspace{-0.5cm}

	The ranking \( \succsim \) satisfies \textbf{Order}, \textbf{Continuity}, \textbf{II Independence}, \textbf{Non Degeneracy}, \textbf{Set Betweenness}, \textbf{Strategic Rationality for Best Likelihood}, \textbf{State Independence} and \textbf{Full Support} if and only if it can be represented by Equations \ref{eq:contmenu1} and \ref{eq:menu1}.

	\begin{equation}
		\mathscr{U}(F)= \sum_{M \in supp \left( F \right)} \left( \sum_{s \in S} F\left( s \right) \left( M \right) \right) \mathcal{U} \left(M ; \ell \left( M_{F} \right) \right) \tag{1}
	\end{equation}

	\begin{equation}
		\begin{aligned}
			\mathcal{U} \left(M ; \ell \left( M_{F} \right) \right) = & \max _{f \in M}\left\{\int_{M_F} u \left( f \left( s \right), \ell \left( M_{F} \right) \right) d p\left(\cdot \mid \ell \left(M_F \right) \right) \right.                \\
			                                                          & \left. +\alpha _{\ell_{M_F}} \int_{M_F} u \left( f \left(s \right) ,  \ell^{\prime}_{M_F} \right) d p\left(\cdot \mid \ell^{\prime}_{M_F} \right)\right\}                 \\
			                                                          & -\max _{f^{\prime} \in M} \alpha _{\ell_{M_F}} \int_{M_F} u\left(f^{\prime}\left(s \right) , \ell^{\prime}_{M_F} \right) d p\left(\cdot \mid \ell^{\prime}_{M_F}\right) .
		\end{aligned} \tag{2}
	\end{equation}

\end{frame}

\end{comment}

\begin{frame}{Conclusion}
	Theory of BDM and motivated updating tested via choices of contingent menus.

	\vfill

	The model predicts same beliefs for same preferences, thus assortativity of beliefs.

	\vfill

	Value of information for individuals with BDM preferences.

\end{frame}

\begin{frame}[noframenumbering,plain]

	\frametitle{References}

	%\nocite{*}
	\bibliography{Others/bib}
	\bibliographystyle{apacite}


\end{frame}

\appendix

\begin{frame}[noframenumbering,plain]{Cost of self-control}

	Identification of \( \alpha_{\ell} \) allows elaboratig on its behavioral meaning

	\vfill

	\[
		\alpha_{\ell} = \frac{\mathcal{U} \left( \left\{f, x \right\}, \ell \right) - \mathcal{U} \left( \left\{f, x^{\prime} \right\}, \ell \right) }{u \left( x , \ell \right) - u \left( x^{\prime} , \ell \right)} .
	\]

	\vfill

	It is the marginal cost of self-control at likelihood \( \ell \).

\end{frame}

\begin{frame}[noframenumbering,plain]{Example: Moral Wiggle Room \citep{danaExploitingMoralWiggle2007}}

	A dictator in a laboratory experiment is endowed with 10 euros.

	\vfill

	She decides how much to transfer to a recipient she is coupled with.

	\vfill

	The transfer is subject to an unknown multiplier, which could be high, medium, or low.

	\vfill

	The experimenter allows the dictator to choose the transfer from various menus conditional on the multiplier's value.

\end{frame}

\begin{frame}[noframenumbering,plain]{Example}
	\begin{table}[H]
		\centering
		\begin{minipage}{0.45\textwidth}
			\centering
			\begin{tabular}{c | c}
				State                          & Actions                                                                                       \\
				\hline
				{\color{bleudefrance} High}    & {\color{bleudefrance}\( h \left\{ 5 \right\} \) } \(+ \left(1-h \right) \left[5,10 \right] \) \\
				{\color{bleudefrance} Medium } & {\color{bleudefrance}\( m \left\{ 5 \right\} \) } \(+ \left(1-m \right) \left[0,5 \right] \)  \\
				Low                            & \( \left[0,3 \right] \)                                                                       \\
			\end{tabular}
			\vspace{0.5cm} % Adjust vertical space between tables and caption
		\end{minipage}\hspace{0.25cm} \pause % Adjust the horizontal space between the tables and the symbol
		\( \succsim \) % Symbol goes here
		\hspace{0.25cm}
		\begin{minipage}{0.45\textwidth}
			\centering
			\begin{tabular}{c | c}
				State                          & Actions                                                                                       \\
				\hline
				{\color{bleudefrance} High}    & {\color{bleudefrance}\( h \left\{ 3 \right\} \) } \(+ \left(1-h \right) \left[5,10 \right] \) \\
				{\color{bleudefrance} Medium } & {\color{bleudefrance}\( m \left\{ 3 \right\} \) } \(+ \left(1-m \right) \left[0,5 \right] \)  \\
				Low                            & \( \left[0,3 \right] \)                                                                       \\
			\end{tabular}
			\vspace{0.5cm} % Adjust vertical space between tables and caption
		\end{minipage}
		%\caption{Set-Betweenness} % Add your caption here
		%\label{tab:sbetweenness}
	\end{table} \pause

	\begin{center}
		\( \Downarrow \)
	\end{center}

	\begin{table}[H]
		\centering
		\begin{minipage}{0.45\textwidth}
			\centering
			\begin{tabular}{c | c}
				State                          & Actions                                                                                       \\
				\hline
				{\color{bleudefrance} High}    & {\color{bleudefrance}\( h \left\{ 5 \right\} \) } \(+ \left(1-h \right) \left[5,10 \right] \) \\
				{\color{bleudefrance} Medium } & {\color{bleudefrance}\( m \left\{ 5 \right\} \) } \(+ \left(1-m \right) \left[0,5 \right] \)  \\
				Low                            & \( \left[0,3 \right] \)                                                                       \\
			\end{tabular}
			\vspace{0.5cm} % Adjust vertical space between tables and caption
		\end{minipage}\hspace{0.25cm} % Adjust the horizontal space between the tables and the symbol
		\( \succsim \) % Symbol goes here
		\hspace{0.25cm}
		\begin{minipage}{0.45\textwidth}
			\centering
			\begin{tabular}{c | c}
				State                          & Actions                                                                                         \\
				\hline
				{\color{bleudefrance} High}    & {\color{bleudefrance}\( h \left\{ 3,5 \right\} \) } \(+ \left(1-h \right) \left[5,10 \right] \) \\
				{\color{bleudefrance} Medium } & {\color{bleudefrance}\( m \left\{ 3,5 \right\} \) } \(+ \left(1-m \right) \left[0,5 \right] \)  \\
				Low                            & \( \left[0,3 \right] \)                                                                         \\
			\end{tabular}
			\vspace{0.5cm} % Adjust vertical space between tables and caption
		\end{minipage}
		\caption{Set-Betweenness} % Add your caption here
		\label{tab:sbetweenness}
	\end{table}
\end{frame}

\begin{frame}[noframenumbering, plain]{Example}
	\begin{table}[H]
		\centering
		\begin{minipage}{0.45\textwidth}
			\centering
			\begin{tabular}{c | c}
				State                          & Actions                                                                                       \\
				\hline
				{\color{bleudefrance} High}    & {\color{bleudefrance}\( h \left\{ 5 \right\} \) } \(+ \left(1-h \right) \left[5,10 \right] \) \\
				{\color{bleudefrance} Medium } & {\color{bleudefrance}\( m \left\{ 5 \right\} \) } \(+ \left(1-m \right) \left[0,5 \right] \)  \\
				Low                            & \( \left[0,3 \right] \)                                                                       \\
			\end{tabular}
		\end{minipage}\hspace{0.25cm} % Adjust the horizontal space between the tables and the symbol
		\( \succ \) % Symbol goes here
		\hspace{0.25cm}
		\begin{minipage}{0.45\textwidth}
			\centering
			\begin{tabular}{c | c}
				State                          & Actions                                                                                       \\
				\hline
				{\color{bleudefrance} High}    & {\color{bleudefrance}\( h \left\{ 7 \right\} \) } \(+ \left(1-h \right) \left[5,10 \right] \) \\
				{\color{bleudefrance} Medium } & {\color{bleudefrance}\( m \left\{ 7 \right\} \) } \(+ \left(1-m \right) \left[0,5 \right] \)  \\
				Low                            & \( \left[0,3 \right] \)                                                                       \\
			\end{tabular}
		\end{minipage}
		%\caption{Set-Betweenness} % Add your caption here
		%\label{tab:srbl}
	\end{table} \pause

	\begin{center}
		\( \Downarrow \)
	\end{center}

	\begin{table}[H]
		\centering
		\begin{minipage}{0.45\textwidth}
			\centering
			\begin{tabular}{c | c}
				State                          & Actions                                                                                       \\
				\hline
				{\color{bleudefrance} High}    & {\color{bleudefrance}\( h \left\{ 5 \right\} \) } \(+ \left(1-h \right) \left[5,10 \right] \) \\
				{\color{bleudefrance} Medium } & {\color{bleudefrance}\( m \left\{ 5 \right\} \) } \(+ \left(1-m \right) \left[0,5 \right] \)  \\
				Low                            & \( \left[0,3 \right] \)                                                                       \\
			\end{tabular}
			\vspace{0.5cm} % Adjust vertical space between tables and caption
		\end{minipage}\hspace{0.25cm} % Adjust the horizontal space between the tables and the symbol
		\( \sim \) % Symbol goes here
		\hspace{0.25cm}
		\begin{minipage}{0.45\textwidth}
			\centering
			\begin{tabular}{c | c}
				State                          & Actions                                                                                         \\
				\hline
				{\color{bleudefrance} High}    & {\color{bleudefrance}\( h \left\{ 5,7 \right\} \) } \(+ \left(1-h \right) \left[5,10 \right] \) \\
				{\color{bleudefrance} Medium } & {\color{bleudefrance}\( m \left\{ 5,7 \right\} \) } \(+ \left(1-m \right) \left[0,5 \right] \)  \\
				Low                            & \( \left[0,3 \right] \)                                                                         \\
			\end{tabular}
			\vspace{0.5cm} % Adjust vertical space between tables and caption
		\end{minipage}
		\caption{Strategic Rationality for Best Likelihood} % Add your caption here
		\label{tab:srbl}
	\end{table}
\end{frame}

\end{document}