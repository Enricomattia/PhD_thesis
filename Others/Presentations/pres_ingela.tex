
\documentclass[usenames,dvipsnames,aspectratio=169,11pt, envcountsect]{beamer}
%handout,
%aspectratio=43

\usepackage{Others/prespreamble}
\usepackage{natbib}
\definecolor{bleudefrance}{rgb}{0.19, 0.55, 0.91}
\linespread{1}
\usepackage{mathrsfs}

% small bibliography
\let\oldthebibliography=\thebibliography
\renewcommand{\thebibliography}[1]{
    \oldthebibliography{#1}
    \setlength{\itemsep}{2pt}
    \tiny
}

\begin{document}

\section{Introduction}

\begin{frame}[noframenumbering,plain]
	\maketitle
	%\begin{figure}[H]
	%\centering
	%\includegraphics[scale=0.4]{Other/Figures/erc.png}
	%\end{figure}

	%\footnotesize{Funding from the European Research Council (ERC) under the European Unions Horizon 2020 research and innovation programme (grant agreement No. 789111 - ERC EvolvingEconomics) is gratefully acknowledged}
\end{frame}
%\frame{\titlepage}

\begin{frame}\frametitle{Motivation}

	Investors overreact to information, consumers refrain from learning about firms' unethical practices, and patients at health risk do not learn about their condition.

	\vfill

	These observations are at odds with expected utility theory.

	\vfill

	Theories of belief-dependent motivations (BDM) explain these phenomena.

	\vfill \pause

	It is unclear whether Bayesian updating and BDM are consistent assumptions.

	\vfill

	\textbf{Question}: how do BDM affect belief updating?

\end{frame}

\begin{frame}\frametitle{This Paper}

	I develop a theory of BDM in a dynamic setting encompassing previous models.

	\vfill

	An individual receives information and takes action under uncertainty.

	\vfill

	She distorts her Bayesian update to satisfy her belief-dependent preferences.

	\vfill

	Maximises a combination of true and distorted belief-dependent expected utility.

	\vfill \pause

	Key observation: Choices over menus allow identyfing the model components.

	\vfill

	\textbf{Main result}: axiomatic representation of BDM preferences and updating rules.

\end{frame}

\begin{frame}{Example: overinvestment and information avoidance}

	A financial investor likes believing the market is in good state.

	\vfill

	She might receive a neutral or bad message about the market state.

	\vfill

	The neutral message does not rule out the good state, the bad is conclusive evidence.

	\vfill

	After the neutral message, she overweights the evidence and overinvests.

	\vfill

	After the bad message, she suffers from the news.

	\vfill

	When choosing whether to receive information, she weighs these two outcomes.

\end{frame}

\begin{frame}\frametitle{Literature}

	\begin{wideitemize}

		\item \textit{Belief-Dependent Motivations.} \cite{eliazCanAnticipatoryFeelings2006}, \cite{benabou2016mindful}, \cite{golmanInformationAvoidance2017}, \cite{battigalliBeliefdependentMotivationsPsychological2022}.

		\vspace{0.3cm}
		\underline{Contribution}: \textbf{Interaction between preferences and belief revision}.

		\item \textit{Decision Theory.} \cite{liangInformationdependentExpectedUtility2017}, \cite{dillenbergerAdditivebeliefbasedPreferences2020} \cite{rommeswinkelPreferenceKnowledge2023}.

		\vspace{0.3cm}
		\underline{Contribution}: \textbf{Belief revision rule}.

		\item \textit{Menu Choice.} \cite{gulTemptationSelfControl2001}, \cite{ozdenorenCompletingStateSpace2002}, \cite{epsteinAxiomaticModelNonBayesian2006}, \cite{epsteinColdFeet2007}.

		\vspace{0.3cm}
		\underline{Contribution}: \textbf{Novel primitive object of choice}.
	\end{wideitemize}

\end{frame}

\section{Body}

\begin{frame}\frametitle{Model}

	Primitives:

	\vfill

	\begin{wideitemize}
		\item
		compact metric outcome set \(X\); \pause
		\item
		finite state spaces \(S\) with prior \( p \in \Delta \left( S \right) \); \pause
		\item
		acts map states to outcomes \(f: S \longrightarrow X \); \pause
		\item experiments map states to finite distribution of messages \( F : S \rightarrow \Delta \left( M \right) \); \pause
		\item Bayesian posterior after observing message \( m \) from experiment \( F \) is \( p_{F,m} \).
	\end{wideitemize}
\end{frame}

\begin{frame}{Utility}

	The individual:

	\vfill

	\begin{wideenumerate}
		\item observes message \( m \) from experiment \( F \) and computes Bayesian posterior \( p_{F,m} \);
		\item distorts the data generating process of \( m \) and develops distorted posterior \( p^{*}_{F,m} \); \pause
		\item chooses act \( f \) to maximise:
	\end{wideenumerate}

	\vfill

	\[
		U \left( f, p_{F,m} \right) = \sum_{s} u \left( f_s, p_{F,m} \right) p_{F,m} \left( s \right) + \pause \alpha_{p_{F,m}} \sum_{s} u \left( f_s, p^{*}_{F,m} \right) p^{*}_{F,m} \left( s \right) .
	\]

\end{frame}


\begin{frame}{Comments}

	\[
		U \left( f, p_{F,m} \right) = \sum_{s} u \left( f_s, p_{F,m} \right) p_{F,m} \left( s \right) + \alpha_{p_{F,m}} \sum_{s} u \left( f_s, p^{*}_{F,m} \right) p^{*}_{F,m} \left( s \right) .
	\]

	\vfill

	The distorted posterior \( p^{*} \) is the Bayesian update under a different experiment.

	\vfill

	It depends on preferences, it is the preferred posterior according to the utility \( u \).

	\vfill

	The probability of states wich are ruled out by the message remains zero.

	\vfill \pause

	If the message induces the preferred \( p^{*} \), then the model collapses to:

	\[
		U \left( f, p^{*}_{F,m} \right) = \sum_{s} u \left( f_s, p^{*}_{F,m} \right) p^{*}_{F,m} \left( s \right) .
	\]

	If \( u \) does not depend on beliefs, it collapses to expected utility.

\end{frame}

\begin{frame}{Interpretation}
	The individual's might:

	\vfill

	\begin{wideitemize}
		\item forget she received a message \citep{benabou2016mindful}; \pause
		\item choose her beliefs \citep{brunnermeierOptimalExpectations2005}; \pause
		\item exhibit motivated inattention \citep{caplinRevealedPreferenceRational2015}; \pause
		\item be optimistic \citep{sharotOptimismBias2011}.
	\end{wideitemize}

	\vfill

	All these interpretations are consistent with the model.

	\vfill

	Distorsion is with respect to the individual's \textbf{subjective beliefs}.

\end{frame}

\begin{frame}{Example: overinvestment and information avoidance}

	The market might be in a high or low state.

	\vfill

	Experiment delivers a neutral \( n \) or bad \( b \) message.

	\vfill

	Preferences are increasing in the posterior on the good state.

	\vfill

	\[ u \left( \cdot , p^{*}_{F,m } \right) = v \left( \cdot, p^{*}_{F, m} \right) - d \left( p^{*}_{F, m} , p_{F, m} \right)
	\] \pause

	\[
		U \left( f, p_{F,n} \right) = \sum_{s} v \left( f_s, p_{F,n} \right) p_{F, n} \left( s \right) + \alpha_{F, n} \left[ \sum_{s} v \left( f_s, p^{*}_{F, m_n} \right) p^{*}_{F,n} \left( s \right) - d(p^{*}_{F,n}, p_{F,n}) \right].
	\] \pause

	\[
		U \left( f, p_{F,b} \right) = \sum_{s} v \left( f_s, p_{F,b} \right) p_{F,b} \left( s \right) .
	\]

\end{frame}

\begin{frame}{Example: Contingent Menus}
	Say the investor can choose wheter to check the status of her portfolio.

	\vfill

	By checking, she gets a signal on the state of the market and can invest or withdraw.

	\vfill

	If she does not check, she gets no information and can do nothing.

	\vfill

	Checking delivers information and induces a \textbf{state contingent menu of choices}.
\end{frame}

\begin{frame}{Example: Temptation}
	When choosing to check, the investor anticipates she will overweight the evidence.

	\vfill

	She knows she will be tempted to act following her distorted beliefs and overinvest.

	\vfill

	She wants to receive information but prefers to commit not to face temptation.

	\vfill \pause

	If the Bayesian posterior coincides with the preferred one, there is no distortion.

	\vfill

	If beliefs are not distorted, there is no temptation.
\end{frame}

\begin{frame}{Model: Contingent Menus}
	\begin{wideitemize}
		\item
		compact metric outcome set \(X\), set of lotteries \(\Delta^{0} \left( X \right)\); \pause
		\item
		finite state spaces \(S\); \pause
		\item
		acts map states to outcomes \(f: S \longrightarrow X \); \pause
		\item
		a closed subset \(m\) of acts is a menu; \pause
		\item
		compact metric set of menus is \(M\), set of lotteries with finite support \(\Delta \left( M \right)\); \pause
		\item
		contingent menus map states to distribution of menus \(F: S \rightarrow \Delta \left( M \right) \); \pause
		\item
		time \(0\) preference on contingent menus.
	\end{wideitemize}
\end{frame}

\begin{frame}{Main Result}

	If the individual does not exhibit temptation at her preferred posterior, then her behaviour is represented by equation \ref{eq:contmenu1} and \ref{eq:menu1}:

	\vfill

	\begin{equation}\label{eq:contmenu1}
		\mathscr{U}(F)= \sum_{m} \left( \sum_{s} F_{s} \left( m \right) \right) \mathcal{U} \left(m ; p_{F,m} \right) \: ;
	\end{equation} \pause

	\begin{equation}\label{eq:menu1}
		\begin{aligned}
			\mathcal{U} \left(m ; p_{F,m} \right) = & \max _{f \in m}\left\{\sum_{s} u \left( f_{s} ; p_{F,m} \right) p_{F,m} \left( s \right) \pause + \alpha _{p_{F,m}} \sum_{s} u \left( f_{s} ; p^{*}_{F,m} \right) p^{*}_{F,m} \left( s \right) \right\} \\ \pause
			                                        & -\max _{f^{\prime} \in m} \alpha _{p_{F,m}} \sum_{s} u\left(f^{\prime}_{s} ; p^{*}_{F,m} \right) p^{*}_{F,m} \left( s \right) \: .
		\end{aligned}
	\end{equation}

	Utility \( u \), beliefs \( p \), distorted beliefs \( p^{*} \) and cost of self-control \( \alpha \) are identified.

\end{frame}

\begin{frame}{Interpretation}

	Choice at period \( 2 \) is described by the following

	\vfill


	\[
		\max _{f \in m} \left[\sum_{s} u \left( f_{s} ; p_{F,m} \right) p_{F,m} \left( s \right) \pause + \alpha _{p_{F,m}} \sum_{s} u \left( f_{s} ; p^{*}_{F,m} \right) p^{*}_{F,m} \left( s \right)  \right].
	\]

	\vfill \pause


	When choosing act \( f \) from menu \( m \) at posterior \( p_{F,m} \), the cost of temptation is

	\vfill

	\[
		\alpha_{p_{F,m}} \left[ \max _{f^{\prime} \in m} \sum_{s} u\left(f^{\prime}_{s} ; p^{*}_{F,m} \right) p^{*}_{F,m} \left( s \right) - \sum_{s} u \left( f_{s} ; p^{*}_{F,m} \right) p^{*}_{F,m} \left( s \right) \right] .
	\]

\end{frame}

\begin{frame}{Conclusion and future directions}
	Theory of BDM and belief updating tested via choices of contingent menus.

	\vfill

	BDM and non Bayesian updating should be considered together.

	\vfill

	The model predicts same beliefs for same preferences, thus assortativity of beliefs.

	\vfill

	Value of information for individuals with BDM preferences.

\end{frame}

\begin{comment}

\begin{frame}\frametitle{Model}

	Primitives:

	\vfill

	\begin{itemize}
		\item
		      compact outcome set \(X\);
		      \pause
		\item
		      finite state spaces \(S\) and prior \( p \in \Delta \left( S \right) \);
		\item Blackwell experiment \( F : S \longrightarrow \Delta \left( M \right) \) where \( m \in M \) is a message;
		\item individual chooses action \(f: S \longrightarrow X \).
	\end{itemize}

	\vfill

	Message \( m \) from experiment \( F \) induces a posterior \( p_{F, m} \).

	\vfill

	Choosing action \( f \) at posterior \( p_{F, m} \) induces utility

	\[
		U \left( f, p_{F, m} \right) = \sum_{s} u \left( f_{s} ; p_{F, m} \right) p_{F, m} \left( s \right) +\alpha _{F, m} \sum_{s} u \left( f_{s} ; p^{*}_{F, m} \right) p^{*}_{F, m} \left( s \right) .
	\]

\end{frame}

\end{comment}

\begin{frame}[noframenumbering,plain]

	\frametitle{References}

	%\nocite{*}
	\bibliography{Others/bib}
	\bibliographystyle{apacite}


\end{frame}

\appendix

\begin{frame}[noframenumbering,plain]{Cost of self-control}

	Identification of \( \alpha_{\ell} \) allows elaboratig on its behavioral meaning

	\vfill

	\[
		\alpha_{\ell} = \frac{\mathcal{U} \left( \left\{f, x \right\}, \ell \right) - \mathcal{U} \left( \left\{f, x^{\prime} \right\}, \ell \right) }{u \left( x , \ell \right) - u \left( x^{\prime} , \ell \right)} .
	\]

	\vfill

	It is the marginal cost of self-control at likelihood \( \ell \).

\end{frame}

\begin{frame}[noframenumbering,plain]{Example: Moral Wiggle Room \citep{danaExploitingMoralWiggle2007}}

	A dictator in a laboratory experiment is endowed with 10 euros.

	\vfill

	She decides how much to transfer to a recipient she is coupled with.

	\vfill

	The transfer is subject to an unknown multiplier, which could be high, medium, or low.

	\vfill

	The experimenter allows the dictator to choose the transfer from various menus conditional on the multiplier's value.

\end{frame}

\begin{frame}[noframenumbering,plain]{Example}
	\begin{table}[H]
		\centering
		\begin{minipage}{0.45\textwidth}
			\centering
			\begin{tabular}{c | c}
				State                          & Actions                                                                                       \\
				\hline
				{\color{bleudefrance} High}    & {\color{bleudefrance}\( h \left\{ 5 \right\} \) } \(+ \left(1-h \right) \left[5,10 \right] \) \\
				{\color{bleudefrance} Medium } & {\color{bleudefrance}\( m \left\{ 5 \right\} \) } \(+ \left(1-m \right) \left[0,5 \right] \)  \\
				Low                            & \( \left[0,3 \right] \)                                                                       \\
			\end{tabular}
			\vspace{0.5cm} % Adjust vertical space between tables and caption
		\end{minipage}\hspace{0.25cm} \pause % Adjust the horizontal space between the tables and the symbol
		\( \succsim \) % Symbol goes here
		\hspace{0.25cm}
		\begin{minipage}{0.45\textwidth}
			\centering
			\begin{tabular}{c | c}
				State                          & Actions                                                                                       \\
				\hline
				{\color{bleudefrance} High}    & {\color{bleudefrance}\( h \left\{ 3 \right\} \) } \(+ \left(1-h \right) \left[5,10 \right] \) \\
				{\color{bleudefrance} Medium } & {\color{bleudefrance}\( m \left\{ 3 \right\} \) } \(+ \left(1-m \right) \left[0,5 \right] \)  \\
				Low                            & \( \left[0,3 \right] \)                                                                       \\
			\end{tabular}
			\vspace{0.5cm} % Adjust vertical space between tables and caption
		\end{minipage}
		%\caption{Set-Betweenness} % Add your caption here
		%\label{tab:sbetweenness}
	\end{table} \pause

	\begin{center}
		\( \Downarrow \)
	\end{center}

	\begin{table}[H]
		\centering
		\begin{minipage}{0.45\textwidth}
			\centering
			\begin{tabular}{c | c}
				State                          & Actions                                                                                       \\
				\hline
				{\color{bleudefrance} High}    & {\color{bleudefrance}\( h \left\{ 5 \right\} \) } \(+ \left(1-h \right) \left[5,10 \right] \) \\
				{\color{bleudefrance} Medium } & {\color{bleudefrance}\( m \left\{ 5 \right\} \) } \(+ \left(1-m \right) \left[0,5 \right] \)  \\
				Low                            & \( \left[0,3 \right] \)                                                                       \\
			\end{tabular}
			\vspace{0.5cm} % Adjust vertical space between tables and caption
		\end{minipage}\hspace{0.25cm} % Adjust the horizontal space between the tables and the symbol
		\( \succsim \) % Symbol goes here
		\hspace{0.25cm}
		\begin{minipage}{0.45\textwidth}
			\centering
			\begin{tabular}{c | c}
				State                          & Actions                                                                                         \\
				\hline
				{\color{bleudefrance} High}    & {\color{bleudefrance}\( h \left\{ 3,5 \right\} \) } \(+ \left(1-h \right) \left[5,10 \right] \) \\
				{\color{bleudefrance} Medium } & {\color{bleudefrance}\( m \left\{ 3,5 \right\} \) } \(+ \left(1-m \right) \left[0,5 \right] \)  \\
				Low                            & \( \left[0,3 \right] \)                                                                         \\
			\end{tabular}
			\vspace{0.5cm} % Adjust vertical space between tables and caption
		\end{minipage}
		\caption{Set-Betweenness} % Add your caption here
		\label{tab:sbetweenness}
	\end{table}
\end{frame}

\begin{frame}[noframenumbering, plain]{Example}
	\begin{table}[H]
		\centering
		\begin{minipage}{0.45\textwidth}
			\centering
			\begin{tabular}{c | c}
				State                          & Actions                                                                                       \\
				\hline
				{\color{bleudefrance} High}    & {\color{bleudefrance}\( h \left\{ 5 \right\} \) } \(+ \left(1-h \right) \left[5,10 \right] \) \\
				{\color{bleudefrance} Medium } & {\color{bleudefrance}\( m \left\{ 5 \right\} \) } \(+ \left(1-m \right) \left[0,5 \right] \)  \\
				Low                            & \( \left[0,3 \right] \)                                                                       \\
			\end{tabular}
		\end{minipage}\hspace{0.25cm} % Adjust the horizontal space between the tables and the symbol
		\( \succ \) % Symbol goes here
		\hspace{0.25cm}
		\begin{minipage}{0.45\textwidth}
			\centering
			\begin{tabular}{c | c}
				State                          & Actions                                                                                       \\
				\hline
				{\color{bleudefrance} High}    & {\color{bleudefrance}\( h \left\{ 7 \right\} \) } \(+ \left(1-h \right) \left[5,10 \right] \) \\
				{\color{bleudefrance} Medium } & {\color{bleudefrance}\( m \left\{ 7 \right\} \) } \(+ \left(1-m \right) \left[0,5 \right] \)  \\
				Low                            & \( \left[0,3 \right] \)                                                                       \\
			\end{tabular}
		\end{minipage}
		%\caption{Set-Betweenness} % Add your caption here
		%\label{tab:srbl}
	\end{table} \pause

	\begin{center}
		\( \Downarrow \)
	\end{center}

	\begin{table}[H]
		\centering
		\begin{minipage}{0.45\textwidth}
			\centering
			\begin{tabular}{c | c}
				State                          & Actions                                                                                       \\
				\hline
				{\color{bleudefrance} High}    & {\color{bleudefrance}\( h \left\{ 5 \right\} \) } \(+ \left(1-h \right) \left[5,10 \right] \) \\
				{\color{bleudefrance} Medium } & {\color{bleudefrance}\( m \left\{ 5 \right\} \) } \(+ \left(1-m \right) \left[0,5 \right] \)  \\
				Low                            & \( \left[0,3 \right] \)                                                                       \\
			\end{tabular}
			\vspace{0.5cm} % Adjust vertical space between tables and caption
		\end{minipage}\hspace{0.25cm} % Adjust the horizontal space between the tables and the symbol
		\( \sim \) % Symbol goes here
		\hspace{0.25cm}
		\begin{minipage}{0.45\textwidth}
			\centering
			\begin{tabular}{c | c}
				State                          & Actions                                                                                         \\
				\hline
				{\color{bleudefrance} High}    & {\color{bleudefrance}\( h \left\{ 5,7 \right\} \) } \(+ \left(1-h \right) \left[5,10 \right] \) \\
				{\color{bleudefrance} Medium } & {\color{bleudefrance}\( m \left\{ 5,7 \right\} \) } \(+ \left(1-m \right) \left[0,5 \right] \)  \\
				Low                            & \( \left[0,3 \right] \)                                                                         \\
			\end{tabular}
			\vspace{0.5cm} % Adjust vertical space between tables and caption
		\end{minipage}
		\caption{Strategic Rationality for Best Likelihood} % Add your caption here
		\label{tab:srbl}
	\end{table}
\end{frame}


\begin{frame}[noframenumbering, plain]{Illustrative Example}

	An investor decides whether to check the state of her portfolio.

	\vfill

	After checking, she decides whether to invest more \( \left( i \right) \) or withdraw any feasible amount of money, which could be high \( \left( \overline{w} \right) \) or low \( \left( w \right) \).

	\vfill

	\begin{table}[H]
		\centering
		\begin{minipage}{0.29\textwidth}

		\end{minipage}\hspace{0.3cm} % Adjust the horizontal space between the tables and the symbol
		% Symbol goes here
		\begin{minipage}{0.29\textwidth}
			\centering
			\begin{tabular}{c | c}
				\multicolumn{2}{c}{\textbf{Check}}                                                                            \\
				State                        & Actions                                                                        \\
				\hline
				{\color{bleudefrance}Good}   & \multirow{2}{*}{{\color{bleudefrance}\( i, \left[ 0, \overline{w} \right] \)}} \\
				{\color{bleudefrance}Normal} &                                                                                \\
				Bad                          & \(  i, \left[0, w \right] \)                                                   \\
			\end{tabular}
			\vspace{0.5cm} % Adjust vertical space between tables and caption
		\end{minipage}\hspace{0.3cm} % Adjust the horizontal space between the tables and the symbol
		% Symbol goes here
		\begin{minipage}{0.29\textwidth}

		\end{minipage}
		%\caption{Commitment under positive prior belief to avoid excessive investment.} % Add your caption here
		%\label{tab:commitment}
	\end{table} \pause

	\vfill

	Upon observing a high amount in it she infers the state of the market is not bad.

	\vfill

	When she sees a low amount, she knows the state of the market is bad.

\end{frame}

\begin{frame}[noframenumbering, plain]{Illustrative Example}

	She can't make any inferences or do anything if she does not check.

	\vfill

	\begin{table}[H]
		\centering
		\begin{minipage}{0.29\textwidth}

		\end{minipage}\hspace{0.3cm} % Adjust the horizontal space between the tables and the symbol
		% Symbol goes here
		\begin{minipage}{0.29\textwidth}
			\centering
			\begin{tabular}{c | c}
				\multicolumn{2}{c}{\textbf{Check}}                                                                            \\
				State                        & Actions                                                                        \\
				\hline
				{\color{bleudefrance}Good}   & \multirow{2}{*}{{\color{bleudefrance}\( i, \left[ 0, \overline{w} \right] \)}} \\
				{\color{bleudefrance}Normal} &                                                                                \\
				Bad                          & \(  i, \left[0, w \right] \)                                                   \\
			\end{tabular}
			\vspace{0.5cm} % Adjust vertical space between tables and caption
		\end{minipage}\hspace{0.7cm} % Adjust the horizontal space between the tables and the symbol
		% Symbol goes here
		\begin{minipage}{0.29\textwidth}
			\centering
			\begin{tabular}{c | c}
				\multicolumn{2}{c}{\textbf{Not Check}} \\
				State  & Actions                       \\
				\hline
				Good   & \multirow{3}{*}{ \( 0 \)}     \\
				Normal &                               \\
				Bad    &                               \\
			\end{tabular}
			\vspace{0.5cm} % Adjust vertical space between tables and caption
		\end{minipage}
		%\caption{Commitment under positive prior belief to avoid excessive investment.} % Add your caption here
		%\label{tab:commitment}
	\end{table}

\end{frame}

\begin{frame}[noframenumbering, plain]{Illustrative Example}

	She could also check and committ not to invest, by delegating to a financial advisor.

	\vfill

	\begin{table}[H]
		\centering
		\begin{minipage}{0.29\textwidth}
			\centering
			\begin{tabular}{c | c}
				\multicolumn{2}{c}{\textbf{Delegate}}                                                                     \\
				State                        & Actions                                                                    \\
				\hline
				{\color{bleudefrance}Good}   & \multirow{2}{*}{{\color{bleudefrance}\( \left[0, \overline{w} \right] \)}} \\
				{\color{bleudefrance}Normal} &                                                                            \\
				Bad                          & \( \left[0, w \right]\)                                                    \\
			\end{tabular}
			\vspace{0.5cm} % Adjust vertical space between tables and caption
		\end{minipage}\hspace{0.5cm} % Adjust the horizontal space between the tables and the symbol
		% Symbol goes here
		\begin{minipage}{0.29\textwidth}
			\centering
			\begin{tabular}{c | c}
				\multicolumn{2}{c}{\textbf{Check}}                                                                            \\
				State                        & Actions                                                                        \\
				\hline
				{\color{bleudefrance}Good}   & \multirow{2}{*}{{\color{bleudefrance}\( i, \left[ 0, \overline{w} \right] \)}} \\
				{\color{bleudefrance}Normal} &                                                                                \\
				Bad                          & \(  i, \left[0, w \right] \)                                                   \\
			\end{tabular}
			\vspace{0.5cm} % Adjust vertical space between tables and caption
		\end{minipage}\hspace{0.5cm} % Adjust the horizontal space between the tables and the symbol
		% Symbol goes here
		\begin{minipage}{0.29\textwidth}
			\centering
			\begin{tabular}{c | c}
				\multicolumn{2}{c}{\textbf{Not Check}} \\
				State  & Actions                       \\
				\hline
				Good   & \multirow{3}{*}{ \( 0 \)}     \\
				Normal &                               \\
				Bad    &                               \\
			\end{tabular}
			\vspace{0.5cm} % Adjust vertical space between tables and caption
		\end{minipage}
		%\caption{Commitment under positive prior belief to avoid excessive investment.} % Add your caption here
		%\label{tab:commitment}
	\end{table}

\end{frame}

\begin{frame}[noframenumbering, plain]{Illustrative Example}

	She anticipates to overweight evidence and invest too much.

	\vfill

	Therefore, she prefers to commit, but also wants to obtain information.

	\vfill

	\begin{table}[H]
		\centering
		\begin{minipage}{0.29\textwidth}
			\centering
			\begin{tabular}{c | c}
				\multicolumn{2}{c}{\textbf{Delegate}}                                                                     \\
				State                        & Actions                                                                    \\
				\hline
				{\color{bleudefrance}Good}   & \multirow{2}{*}{{\color{bleudefrance}\( \left[0, \overline{w} \right] \)}} \\
				{\color{bleudefrance}Normal} &                                                                            \\
				Bad                          & \( \left[0, w \right]\)                                                    \\
			\end{tabular}
			\vspace{0.5cm} % Adjust vertical space between tables and caption
		\end{minipage}\hspace{0.3cm} % Adjust the horizontal space between the tables and the symbol
		\( \succ \) % Symbol goes here
		\begin{minipage}{0.29\textwidth}
			\centering
			\begin{tabular}{c | c}
				\multicolumn{2}{c}{\textbf{Check}}                                                                            \\
				State                        & Actions                                                                        \\
				\hline
				{\color{bleudefrance}Good}   & \multirow{2}{*}{{\color{bleudefrance}\( i, \left[ 0, \overline{w} \right] \)}} \\
				{\color{bleudefrance}Normal} &                                                                                \\
				Bad                          & \(  i, \left[0, w \right] \)                                                   \\
			\end{tabular}
			\vspace{0.5cm} % Adjust vertical space between tables and caption
		\end{minipage}\hspace{0.3cm} % Adjust the horizontal space between the tables and the symbol
		\( \succ \) % Symbol goes here
		\begin{minipage}{0.29\textwidth}
			\centering
			\begin{tabular}{c | c}
				\multicolumn{2}{c}{\textbf{Not Check}} \\
				State  & Actions                       \\
				\hline
				Good   & \multirow{3}{*}{ \( 0 \)}     \\
				Normal &                               \\
				Bad    &                               \\
			\end{tabular}
			\vspace{0.5cm} % Adjust vertical space between tables and caption
		\end{minipage}
		\caption{Commitment under positive prior belief to avoid excessive investment.} % Add your caption here
		\label{tab:commitment}
	\end{table}

	\vfill

	"Cognitive" non-Bayesian updating \citep{epsteinAxiomaticModelNonBayesian2006} cannot rationalise this behaviour.

\end{frame}

\begin{frame}[noframenumbering, plain]{Illustrative Example}

	If the investor expects the state of the market to be bad, she prefers not to check the portfolio at all to avoid receiving unpleasant information.

	\vfill

	\begin{table}[H]
		\centering
		\begin{minipage}{0.29\textwidth}
			\centering
			\begin{tabular}{c | c}
				\multicolumn{2}{c}{\textbf{Not Check}} \\
				State  & Actions                       \\
				\hline
				Good   & \multirow{3}{*}{ \( 0 \)}     \\
				Normal &                               \\
				Bad    &                               \\
			\end{tabular}
			\vspace{0.5cm} % Adjust vertical space between tables and caption
		\end{minipage}\hspace{0.3cm} % Adjust the horizontal space between the tables and the symbol
		\( \succ \) % Symbol goes here
		\begin{minipage}{0.29\textwidth}
			\centering
			\begin{tabular}{c | c}
				\multicolumn{2}{c}{\textbf{Delegate}}                                                                     \\
				State                        & Actions                                                                    \\
				\hline
				{\color{bleudefrance}Good}   & \multirow{2}{*}{{\color{bleudefrance}\( \left[0, \overline{w} \right] \)}} \\
				{\color{bleudefrance}Normal} &                                                                            \\
				Bad                          & \( \left[0, w \right]\)                                                    \\
			\end{tabular}
			\vspace{0.5cm} % Adjust vertical space between tables and caption
		\end{minipage}\hspace{0.3cm} % Adjust the horizontal space between the tables and the symbol
		\( \succ \) % Symbol goes here
		\begin{minipage}{0.29\textwidth}
			\centering
			\begin{tabular}{c | c}
				\multicolumn{2}{c}{\textbf{Check}}                                                                            \\
				State                        & Actions                                                                        \\
				\hline
				{\color{bleudefrance}Good}   & \multirow{2}{*}{{\color{bleudefrance}\( i, \left[ 0, \overline{w} \right] \)}} \\
				{\color{bleudefrance}Normal} &                                                                                \\
				Bad                          & \(  i, \left[0, w \right] \)                                                   \\
			\end{tabular}
			\vspace{0.5cm} % Adjust vertical space between tables and caption
		\end{minipage}
		\caption{Information avoidance under negative prior beliefs, "ostrich effect".} % Add your caption here
		\label{tab:oistrich}
	\end{table}

	Both excessive trading and the ostrich effect constitutes empirical puzzles in finance \citep{danielOverconfidentInvestorsPredictable2015,golmanInformationAvoidance2017}.

\end{frame}


\end{document}