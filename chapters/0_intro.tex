\chapter*{Introduction}\label{ch:intro} % no number
\addcontentsline{toc}{chapter}{Introduction}% put it in the ToC

This thesis studies principles underlying individual behaviour, information processing, and resource allocation. It focuses on foundational issues in behavioural economics, where the concepts investigated in this thesis are introduced. Behavioural economics—which examines variations in individual behaviour, belief revision, and normative judgement—has historically been characterized by a reduced-form approach \citep{spieglerBehavioralEconomicsAtheoretical2019}. This approach explains empirical observations through reinterpretations of classical theoretical constructs that intuitively capture relevant psychological mechanisms. In contrast, I argue that more nuanced treatments are necessary, requiring the development of novel theoretical tools rather than merely reinterpreting existing ones. As a result, I derive distinctions within the behavioural phenomena under study that are difficult to discern without explicit modelling.

The chapters appear in the chronological order of their conception. In Chapter \ref{ch:univ}, I study individuals who universalise their behaviour—that is, they consider what would happen if everyone behaved as they do, under different interpretations of this notion. In Chapter \ref{ch:meritocracy}, I examine criteria for the allocation of resources in society that are responsive to considerations of merit and individual responsibility. In Chapter \ref{ch:bdp}, I explore how individuals with preferences over their beliefs behave and interpret new information. Common to all three chapters is the use of the axiomatic method. This approach is natural given the aims of the thesis. In the words of Debreu: \textquote{\textit{Allegiance to rigour dictates the axiomatic form of the analysis where the theory, in the strict sense, is logically entirely disconnected from its interpretations}} \citep[p. x]{debreuTheoryValueAxiomatic1959}. Given the conceptual complexity of the topics at hand, the axiomatic method ensures that the logical development of the theory is not contaminated by its interpretations, leading to \textquote{\textit{a deeper understanding of the problem}} \citep[p. x]{debreuTheoryValueAxiomatic1959}.

In all three chapters, I first introduce a syntax that allows me to describe the objects of interest. The principles under study represent particular instances of these objects. I then specify properties that elements of the syntax must satisfy to be equivalent to the object of interest. In particular, I characterise these objects as the unique elements satisfying the relevant properties. This method is useful because it allows one to express a theory of individual behaviour, belief revision, or resource allocation using a few logically consistent conditions. These conditions serve a dual purpose: first, they can be normatively evaluated—whether they are conditions one might wish to respect when acting, processing information, or distributing goods. Second, they provide testable implications. If an individual’s observed behaviour violates these conditions, then the theory is an inadequate description of that individual’s behaviour. Each chapter includes illustrative applications of the theory in canonical economic settings. I now turn to a brief description of the chapters.

\emph{Chapter \ref{ch:univ}: A Foundation for Universalisation in Games.} The first chapter studies individuals with preferences for universalisation—that is, they consider what would happen if everyone were to act as they do. Universalisation has been shown to have evolutionary foundations, to align with observed behaviour, and to lead to desirable allocations under various normative frameworks. Existing models, such as Homo Moralis preferences \citep{algerHomoMoralisPreference2013} and Kantian equilibrium \citep{roemer2019cooperate}, lack choice-theoretic foundations, limiting their generalisability. To address this, I develop an axiomatic model characterising preferences for universalisation. The main challenge is that universalisation is a non-consequentialist attitude, which is difficult to capture using standard choice-theoretic tools. A key behavioural prediction of my model is that the independence axiom holds only among actions that are universalised in equivalent ways. My framework unifies previous models, introduces a broader class of universalisation preferences, and offers guidance for empirical studies.

\emph{Chapter \ref{ch:meritocracy}: Meritocracy as an End and as a Means.} The second chapter studies the concept of meritocracy, widely discussed both publicly and in the economics and philosophy literature. An allocation is meritocratic if more meritorious individuals obtain more rewarding outcomes. Each instance of meritocracy is characterised by two components: a merit criterion, which determines what counts as meritorious behaviour, and a reward criterion, which specifies which outcomes are more rewarding. By examining whether the allocation choices of impartial spectators align with particular merit and reward criteria, one can test the extent to which individuals adhere to different meritocratic principles. I consider two motivations for supporting meritocracy: rewarding merit as intrinsically fair—interpreting meritocracy as an end—and using meritocracy as an instrument to achieve other goals, such as efficiency—thus treating it as a means. I show that these two justifications are equivalent in terms of the rules they imply. Different assumptions about the merit and reward criteria accommodate various instances of meritocracy. I characterise and examine two meritocratic principles found in the literature: Pareto meritocracy, in which merit derives from generating a Pareto improvement, and proportional meritocracy, in which consumption increases proportionally with effort. I conclude by distinguishing meritocracy from responsibility-sensitive egalitarianism, as in \citep{fleurbaey2008fairness}.

\emph{Chapter \ref{ch:bdp}: Identifying Belief-dependent Preferences.} The third chapter studies individuals whose well-being—their preferences over outcomes—is directly shaped by their beliefs. Such belief-dependent preferences explain a range of behaviours that deviate from expected utility theory. A growing body of evidence suggests that individuals selectively avoid or distort information, consistent with a preference for holding particular beliefs \citep{golmanInformationAvoidance2017}. When beliefs influence preferences over outcomes, belief formation itself may be endogenously shaped by those preferences. This interdependence complicates the task of inferring tastes and beliefs from choice data. The main contribution of the chapter is to present a model of belief-dependent preferences combined with non-Bayesian updating, and to provide choice data sufficient to test and identify the model’s components. I introduce a novel form of choice data, which generalises the notion of a menu in the menu-choice literature, and introduce axioms over preferences on such menus. To clarify the contrast with existing approaches, consider \cite{brunnermeierOptimalExpectations2005}, where individuals choose their beliefs, balancing belief-based utility with material payoffs. While technically equivalent to standard models, this framework departs from prior decision-theoretic foundations by assuming endogenous belief choice. As noted by \cite{eliazCanAnticipatoryFeelings2006} and \cite{spieglerBehavioralEconomicsAtheoretical2019}, this assumption complicates the interpretation and testability of the model. In contrast, my model uses standard tools from choice theory to identify the behavioural implications of belief dependence and provides clear conditions under which the theory can be falsified.

\bibliographystyle{apacite}  % or another  style
\bibliography{references} % .bib file goes in ./bib/